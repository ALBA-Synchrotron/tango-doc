
\chapter{The property file syntax}


\section{Property file usage}

A property file is a file where you store all the property(ies) related
to device(s) belonging to a specific device server process. In this
file, one can find:
\begin{itemize}
\item Which device(s) has to be created for each Tango class embedded in
the device server process
\item Device(s) properties
\item Device(s) attribute properties
\end{itemize}
This type of file is not required by a Tango control system. These
informations are stored in the Tango database and having them also
in a file could generate some data duplication issues. Nevertheless,
in some cases, it could very very helpful to generate this type of
file. These cases are:
\begin{enumerate}
\item If you want to run a device server process on a host which does not
have access to the Tango control system database. In such a case,
the user can generate the file from the database content and run the
device server process using this file as database (-file option of
device server process)
\item In case of massive property changes where no tool will be more adapted
than your favorite text editor. In such a case, the user can generate
a file from the database content, change/add/modify file contents
using his favorite tool and then reload file content into the database.
\end{enumerate}
Jive\cite{Jive doc} is the tool provided to generate and load a property
file. To generate a device server process properties file, select
your device server process in the \textquotedbl{}Server\textquotedbl{}
tab, right click and select \textquotedbl{}Save Server Data\textquotedbl{}.
A file selection window pops up allowing you to choose your file name
and path. To reload a file in the Tango database, click on \textquotedbl{}File\textquotedbl{}
then \textquotedbl{}Load Property File\textquotedbl{}. 


\section{Property file syntax}


\begin{minted}[linenos]{cpp}
\textbf{1 }#---------------------------------------------------------
2 # SERVER TimeoutTest/manu, TimeoutTest device declaration
3 #---------------------------------------------------------
4 
5 TimeoutTest/manu/DEVICE/TimeoutTest: "et/to/01",\ 
6                                      "et/to/02",\ 
7                                      "et/to/03"
8 
9 
10 # --- et/to/01 properties
11 
12 et/to/01->StringProp: Property
13 et/to/01->ArrayProp: 1,\ 
14                      2,\ 
15                      3
16 et/to/01->attr_min_poll_period: TheAttr,\ 
17                                 1000
18 et/to/01->AnotherStringProp: "A long string"
19 et/to/01->ArrayStringProp: "the first prop",\ 
20                            "the second prop"
21 
22 # --- et/to/01 attribute properties
23 
24 et/to/01/TheAttr->display_unit: 1.0
25 et/to/01/TheAttr->event_period: 1000
26 et/to/01/TheAttr->format: %4d
27 et/to/01/TheAttr->min_alarm: -2.0
28 et/to/01/TheAttr->min_value: -5.0
29 et/to/01/TheAttr->standard_unit: 1.0
30 et/to/01/TheAttr->__value: 111
31 et/to/01/BooAttr->event_period: 1000doc_url
32 et/to/01/TestAttr->display_unit: 1.0
33 et/to/01/TestAttr->event_period: 1000
34 et/to/01/TestAttr->format: %4d
35 et/to/01/TestAttr->standard_unit: 1.0
36 et/to/01/DbAttr->abs_change: 1.1
37 et/to/01/DbAttr->event_period: 1000
38
39 CLASS/TimeoutTest->InheritedFrom:   Device_4Impl
40 CLASS/TimeoutTest->doc_url:   "http://www.esrf.fr/some/path"
\end{minted}


Line 1 - 3: Comments. Comment starts with the '\#' character

Line 4: Blank line

Line 5 - 7: Devices definition. \textquotedbl{}DEVICE\textquotedbl{}
is the keyword to declare a device(s) definition sequence. The general
syntax is:\begin{center}<DS name>/<inst name>/DEVICE/<Class name>:
dev1,dev2,dev3\end{center}Device(s) name can follow on next line
if the last line character is '\textbackslash{}' (see line 5,6). The
'\textquotedbl{}' characters around device name are generated by the
Jive tool and are not mandatory.

Line 12: Device property definition. The general device property syntax
is \begin{center}<device name>\textbf{->}<property name>: <property
value>\end{center}In case of array, the array element delimiter is
the character ','. Array definition can be splitted on several lines
if the last line character is '\textbackslash{}'. Allowed characters
after the ':' delimiter are space, tabulation or nothing.

Line 13 - 15 and 16 - 17: Device property (array)

Line 18: A device string property with special characters (spaces).
The '\textquotedbl{}' character is used to delimit the string

Line 24 - 37: Device attribute property definition. The general device
attribute property syntax is \begin{center}<device name>/<attribute
name>\textbf{->}<property name>: <property value>\end{center}Allowed
characters after the ':' delimiter are space, tabulation or nothing.

Line 39 - 40: Class property definition. The general class property
syntax is \begin{center}CLASS/<class name>\textbf{->}<property name>:
<property value>\end{center}\textquotedbl{}CLASS\textquotedbl{} is
the keyword to declare a class property definition. Allowed characters
after the ':' delimiter are space, tabulation or nothing. On line
40, the '\textquotedbl{}' characters around the property value are
mandatory due to the '/' character contains in the property value.
