
\chapter{The notifd2db utility}


\section{The notifd2db utility usage (For Tango releases lower than 8)}

The notifd2db utility is used to pass to Tango the necessary information
for the Tango servers or clients to build connection with the CORBA
notification service. Its usage is:\begin{center}notifd2db {[}notifd2db\_IOR\_file{]}
{[}host{]} {[}-o Device\_server\_database\_file\_name{]} {[}-h{]}\end{center}



The {[}notifd2db\_IOR\_file{]} parameter is used to specify the file
name used by the notification service to store its main IOR. This
parameter is not mandatoty. Its default value is /tmp/rdfact.ior.
The {[}host{]} parameter is ued to specify on which host the notification
service should be exported. The default value is the host on which
the command is run. The {[}-o Device\_server\_database\_file\_name{]}
is used in case of event and device server started with the file as
database (the -file device server command line option). The file name
used here must be the file name used by the device server in its -file
option. The {[}-h{]} option is just to display an help message. Notifd2db
utility usage example:\begin{center}notifd2db\end{center} to register
notification service on the current host using the default notifictaion
service IOR file name. \begin{center}notifd C:\textbackslash{}Temp\textbackslash{}nd.ior\end{center}
to register a notification service with IOR file named C:\textbackslash{}Temp\textbackslash{}nd.ior.\begin{center}

notifd -o /var/my\_ds\_file.res\end{center} to register notification
service in the /var/my\_ds\_file.res file used by a device server
started with the device server -file command line option.
