\appendix

\chapter{Reference part\label{cha:Reference-part}}

\textbf{This chapter is only part of the TANGO device server reference
guide. To get reference documentation about the C++ library classes,
see \cite{TANGO_ref_man}. To get reference documentation about the
Java classes, also see \cite{TANGO_ref_man}.}

\section{Device parameter}

A black box, a device description field, a device state and status
are associated with each TANGO device.

\subsection{The device black box}

The device black box\index{black-box} is managed as a circular buffer.
It is possible to tune the buffer depth via a device property. This
property name is \begin{center}device name->blackbox\_depth\end{center}
A default value is hard-coded to 50 if the property is not defined.
This black box depth property is retrieved from the Tango property
database during the device creation phase.

\subsection{The device description field}

There are two ways to intialise the device description\index{description}
field.
\begin{itemize}
\item At device creation time. Some constructors of the DeviceImpl class
supports this field as parameter. If these constructor are not used,
the device description field is set to a default value which is \emph{A
Tango device}.
\item With a property. A description field defines with this method overrides
a device description defined at construction time. The property name
is \begin{center}device name->description\end{center}
\end{itemize}

\subsection{The device state and status}

Some constructors of the DeviceImpl class allows the initialisation
of device state\index{state} and/or status\index{status} or device
creation time. If these fields are not defined, a default value is
applied. The default state is Tango::UNKOWN, the default status is
\emph{Not Initialised.}

\subsection{The device polling\label{subsec:The-device-polling-prop}}

Seven device properties allow the polling tunning. These properties
are described in the following table 

\vspace{0.3cm}

\begin{center}
\begin{longtable}{|c|c|c|}
\hline 
Property name & property rule & default value\tabularnewline
\hline 
\hline 
poll\_ring\_depth & Polling buffer depth & 10\tabularnewline
\hline 
cmd\_poll\_ring\_depth & Cmd polling buffer depth & \tabularnewline
\hline 
attr\_poll\_ring\_depth & Attr polling buffer depth & \tabularnewline
\hline 
poll\_old\_factor & \textquotedbl{}Data too old\textquotedbl{} factor & 4\tabularnewline
\hline 
min\_poll\_period & Minimun polling period & \tabularnewline
\hline 
cmd\_min\_poll\_period & Min. polling period for cmd & \tabularnewline
\hline 
attr\_min\_poll\_period & Min. polling period for attr & \tabularnewline
\hline 
\end{longtable}
\par\end{center}

\vspace{0.3cm}

The rule of the poll\_ring\_depth\index{poll-ring-depth} property
is obvious. It defines the polling ring depth for all the device polled
command(s) and attribute(s). Nevertheless, when filling the polling
buffer via the fill\_cmd\_polling\_buffer()\index{fill-cmd-polling-buffer}
(or fill\_attr\_polling\_buffer()\index{fill-attr-polling-buffer})
method, it could be helpfull to define specific polling ring depth
for a command (or an attribute). This is the rule of the cmd\_poll\_ring\_depth\index{cmd-poll-ring-depth}
and attr\_poll\_ring\_depth\index{attr-poll-ring-depth} properties.
For each polled object with specific polling depth (command or attribute),
the syntax of this property is the object name followed by the ring
depth (ie State,20,Status,15). If one of these properties is defined,
for the specific command or attribute, it will overwrite the value
set by the poll\_ring\_depth property. The poll\_old\_factor\index{poll-old-factor}
property allows the user to tune how long the data recorded in the
polling buffer are valid. Each time some data are read from the polling
buffer, a check is done between the date when the data were recorded
in the polling buffer and the date when the user request these data.
If the interval is greater than the object polling period multiply
by the value of the poll\_old\_factor factory, an exception is returned
to the caller. These two properties are defined at device level and
therefore, it is not possible to tune this parameter for each polled
object (command or attribute). The last 3 properties are dedicated
to define a polling period minimum threshold. The property min\_poll\_period\index{min-poll-period}
defines in (mS) a device minimum polling period. Property cmd\_min\_poll\_period\index{cmd-min-poll-period}
defines (in mS) a minimum polling period for a specific command. The
syntax of this property is the command name followed by the minimum
polling period (ie MyCmd,400). Property attr\_min\_poll\_period\index{attr-min-poll-period}
defines (in mS) a minimum polling period for a specific attribute.
The syntax of this property is the attribute name followed by the
minimum polling period (ie MyAttr,600). These two properties has a
higher priority than the min\_poll\_period property. By default these
three properties are not defined mening that there is no minimun polling
period.

Four other properties are used by the Tango core classes to manage
the polling thread. These properties are :
\begin{itemize}
\item polled\_cmd to memorize the name of the device polled command
\item polled\_attr to memorize the name of the device polled attribute
\item non\_auto\_polled\_cmd to memorize the name of the command which shoule
not be polled automatically at the first request
\item non\_auto\_polled\_attr to memorize the name of the attribute which
should not be polled automatically at the first request
\end{itemize}
You don't have to change these properties values by yourself. They
are automatically created/modified/deleted by Tango core classes.

\subsection{The device logging}

The Tango Logging Service (TLS) uses device properties to control
device logging at startup (static configuration). These properties
are described in the following table 

\vspace{0.3cm}

\begin{center}
\begin{longtable}{|c|c|c|}
\hline 
Property name & property rule & default value\tabularnewline
\hline 
\hline 
logging\_level & Initial device logging level & WARN\tabularnewline
\hline 
logging\_target & Initial device logging target & No default\tabularnewline
\hline 
logging\_rft & Logging rolling file threshold & 20 Mega bytes\tabularnewline
\hline 
logging\_path & Logging file path & \multicolumn{1}{c|}{/tmp/tango-<logging name> or}\tabularnewline
 &  & C:/tango-<logging name> (Windows)\tabularnewline
\hline 
\end{longtable}
\par\end{center}

\vspace{0.3cm}

\begin{itemize}
\item The logging\_level\index{logging-level} property controls the initial
logging level of a device. Its set of possible values is: \textquotedbl{}OFF\textquotedbl{},
\textquotedbl{}FATAL\textquotedbl{}, \textquotedbl{}ERROR\textquotedbl{},
\textquotedbl{}WARN\textquotedbl{}, \textquotedbl{}INFO\textquotedbl{}
or \textquotedbl{}DEBUG\textquotedbl{}. This property is overwritten
by the verbose command line option (-v).
\item The logging\_target\index{logging-target} property is a multi-valued
property containing the initial target list. Each entry must have
the following format: target\_type::target\_name (where target\_type
is one of the supported target types and target\_name, the name of
the target). Supported target types are: \emph{console}, \emph{file}
and \emph{device}. For a device target, target\_name must contain
the name of a log consumer device (as defined in \ref{sec:Tango-log-consumer}).
For a file target, target\_name is the name of the file to log to.
If omitted the device's name is used to build the file name (domain\_family\_member.log).
Finally, target\_name is ignored in the case of a console target.
The TLS does not report any error occurred while trying to setup the
initial targets. 

\begin{itemize}
\item Logging\_target property example : \\
\\
logging\_target = {[} \textquotedbl{}console\textquotedbl{}, \textquotedbl{}file\textquotedbl{},
\textquotedbl{}file::/home/me/mydevice.log\textquotedbl{}, \textquotedbl{}device::tmp/log/1\textquotedbl{}{]}\\
\\
In this case, the device will automatically logs to the standard output,
to its default file (which is something like domain\_family\_member.log),
to a file named mydevice.log and located in /home/me. Finally, the
device logs are also sent to a log consumer device named tmp/log/1. 
\end{itemize}
\item The logging\_rft\index{logging-rft} property specifies the rolling
file threshold (rft), of the device's file targets. This threshold
is expressed in Kb. When the size of a log file reaches the so-called
rolling-file-threshold (rft), it is backuped as \textquotedbl{}\emph{current\_log\_file\_name}\textquotedbl{}
+ \textquotedbl{}\emph{\_1}\textquotedbl{} and a new current\_log\_file\_name
is opened. Obviously, there is only one backup file at a time (i.e.
any existing backup is destroyed before the current log file is backuped).
The default threshold is 20 Mb, the minimum is 500 Kb and the maximum
is 1000 Mb.
\item The logging\_path\index{logging-path} property overwrites the TANGO\_LOG\_PATH\index{TANGO-LOG-PATH}
environment variable. This property can only be applied to a DServer
class device and has no effect on other devices.
\end{itemize}

\section{Device attribute}

Attribute are configured with two kind of parameters: Parameters hard-coded
in source code and modifiable parameters

\subsection{Hard-coded device attribute parameters}

Seven attribute parameters are defined at attribute\index{attribute}
creation time in the Tango class source code. Obviously, these parameters
are not modifiable except with a new source code compilation. These
parameters are 

\vspace{0.3cm}

\begin{center}
\begin{longtable}{|c|c|}
\hline 
Parameter name & Parameter description\tabularnewline
\hline 
\hline 
name & Attribute name\tabularnewline
\hline 
data\_type & Attribute data type\tabularnewline
\hline 
data\_format & Attribute data format\tabularnewline
\hline 
writable\index{writable} & Attribute read/write type\tabularnewline
\hline 
max\_dim\_x & Maximum X dimension\tabularnewline
\hline 
max\_dim\_y & Maximum Y dimension\tabularnewline
\hline 
writable\_attr\_name\index{writable-attr-name} & Associated write attribute\tabularnewline
\hline 
level\index{level} & Attribute display level\tabularnewline
\hline 
root\_attr\_name & Root attribute name\tabularnewline
\hline 
\end{longtable}
\par\end{center}

\vspace{0.3cm}


\subsubsection{The Attribute data type\index{data-type}}

Thirteen data types are supported. These data types are
\begin{itemize}
\item Tango::DevBoolean 
\item Tango::DevShort
\item Tango::DevLong
\item Tango::DevLong64
\item Tango::DevFloat
\item Tango::DevDouble
\item Tango::DevUChar
\item Tango::DevUShort
\item Tango::DevULong
\item Tango::DevULong64
\item Tango::DevString
\item Tango::DevState
\item Tango::DevEncoded
\end{itemize}

\subsubsection{The attribute data format\index{data-format}}

Three data format are supported for attribute

\vspace{0.3cm}

\begin{center}
\begin{longtable}{|c|c|}
\hline 
Format & Description\tabularnewline
\hline 
\hline 
Tango::SCALAR\index{SCALAR} & The attribute value is a single number\tabularnewline
\hline 
Tango::SPECTRUM\index{SPECTRUM} & The attribute value is a one dimension number\tabularnewline
\hline 
Tango::IMAGE\index{IMAGE} & The attribute value is a two dimension number\tabularnewline
\hline 
\end{longtable}
\par\end{center}

\vspace{0.3cm}


\subsubsection{The max\_dim\_x\index{max-dim-x} and max\_dim\_y\index{max-dim-y}
parameters}

These two parameters defined the maximum size for attributes of the
SPECTRUM and IMAGE data format.

\vspace{0.3cm}

\begin{center}
\begin{longtable}{|c|c|c|}
\hline 
data format & max\_dim\_x & max\_dim\_y\tabularnewline
\hline 
\hline 
Tango::SCALAR & 1 & 0\tabularnewline
\hline 
Tango::SPECTRUM & User Defined & 0\tabularnewline
\hline 
Tango::IMAGE & User Defined & User Defined\tabularnewline
\hline 
\end{longtable}
\par\end{center}

\vspace{0.3cm}

For attribute of the Tango::IMAGE data format, all the data are also
returned in a one dimension array. The first array is value{[}0{]},{[}0{]},
array element X is value{[}0{]},{[}X-1{]}, array element X+1 is value{[}1{]}{[}0{]}
and so forth.

\subsubsection{The attribute read/write type}

Tango supports four kind of read/write attribute which are :
\begin{itemize}
\item Tango::READ\index{READ} for read only attribute
\item Tango::WRITE\index{WRITE} for writable attribute
\item Tango::READ\_WRITE\index{READ-WRITE} for attribute which can be read
and write
\item Tango::READ\_WITH\_WRITE\index{READ-WITH-WRITE} for a readable attribute
associated to a writable attribute (For a power supply device, the
current really generated is not the wanted current. To handle this,
two attributes are defined which are \emph{generated\_current} and
\emph{wanted\_current}. The \emph{wanted\_current} is a Tango::WRITE
attribute. When the \emph{generated\_current} attribute is read, it
is very convenient to also get the \emph{wanted\_current} attribute.
This is exactly what the Tango::READ\_WITH\_WRITE attribute is doing)
\end{itemize}
When read, attribute values are always returned within an array even
for scalar attribute. The length of this array and the meaning of
its elements is detailed in the following table for scalar attribute.

\vspace{0.3cm}

\begin{center}
\begin{longtable}{|c|c|c|c|}
\hline 
Name & Array length & Array{[}0{]} & Array{[}1{]}\tabularnewline
\hline 
\hline 
Tango::READ & 1 & Read value & \tabularnewline
\hline 
Tango::WRITE & 1 & Last write value & \tabularnewline
\hline 
Tango::READ\_WRITE & 2 & Read value & Last write value\tabularnewline
\hline 
Tango::READ\_WITH\_WRITE  & 2 & Read value & Associated attributelast write value\tabularnewline
\hline 
\end{longtable}
\par\end{center}

\vspace{0.3cm}

When a spectrum or image attribute is read, it is possible to code
the device class in order to send only some part of the attribute
data (For instance only a Region Of Interest for an image) but never
more than what is defined by the attribute configuration parameters
max\_dim\_x and max\_dim\_y. The number of data sent is also transferred
with the data and is named \textbf{dim\_x}\index{dim-x} and \textbf{dim\_y}\index{dim-y}.
When a spectrum or image attribute is written, it is also possible
to send only some of the attribute data but always less than max\_dim\_x
for spectrum and max\_dim\_x {*} max\_dim\_y for image. The following
table describe how data are returned for spectrum attribute. dim\_x
is the data size sent by the server when the attribute is read and
dim\_x\_w is the data size used during the last attribute write call.

\vspace{0.3cm}

\begin{center}
\begin{longtable}{|c|c|c|c|}
\hline 
Name & Array length & Array{[}0->dim\_x-1{]} & Array{[}dim\_x -> dim\_x + dim\_x\_w -1{]}\tabularnewline
\hline 
\hline 
Tango::READ & dim\_x & Read values & \tabularnewline
\hline 
Tango::WRITE & dim\_x\_w & Last write values & \tabularnewline
\hline 
Tango::READ\_WRITE & dim\_x + dim\_x\_w & Read value & Last write values\tabularnewline
\hline 
Tango::READ\_WITH\_WRITE  & dim\_x + dim\_x\_w & Read value & Associated attributelast write values\tabularnewline
\hline 
\end{longtable}
\par\end{center}

\vspace{0.3cm}

The following table describe how data are returned for image attribute.
dim\_r is the data size sent by the server when the attribute is read
(dim\_x {*} dim\_y) and dim\_w is the data size used during the last
attribute write call (dim\_x\_w {*} dim\_y\_w).

\vspace{0.3cm}

\begin{center}
\begin{longtable}{|c|c|c|c|}
\hline 
Name & Array length & Array{[}0->dim\_r-1{]} & Array{[}dim\_r-> dim\_r + dim\_w -1{]}\tabularnewline
\hline 
\hline 
Tango::READ & dim\_r & Read values & \tabularnewline
\hline 
Tango::WRITE & dim\_w & Last write values & \tabularnewline
\hline 
Tango::READ\_WRITE & dim\_r + dim\_w & Read value & Last write values\tabularnewline
\hline 
Tango::READ\_WITH\_WRITE  & dim\_r + dim\_w & Read value & Associated attributelast write values\tabularnewline
\hline 
\end{longtable}
\par\end{center}

\vspace{0.3cm}

Until a write operation has been performed, the last write value is
initialized to \emph{0} for scalar attribute of the numeriacal type,
to \emph{\textquotedbl{}Not Initialised\textquotedbl{}} for scalar
string attribute and to \emph{true} for scalar boolean attribute.
For spectrum or image attribute, the last write value is initialized
to an array of one element set to \emph{0} for numerical type, to
an array of one element set to \emph{true} for boolean attribute and
to an array of one element set to \textquotedbl{}\emph{Not initialized}\textquotedbl{}
for string attribute

\subsubsection{The associated write attribute parameter}

This parameter has a meaning only for attribute with a Tango::READ\_WITH\_WRITE
read/write type. This is the name of the associated write attribute.

\subsubsection{The attribute display level\index{level} parameter\label{display level}}

This parameter is only an help for graphical application. It is a
C++ enumeration starting at 0. The code associated with each attribute
display level is defined in the following table (Tango::DispLevel\index{DispLevel}).

\vspace{0.3cm}

\begin{center}
\begin{longtable}{|c|c|}
\hline 
name & Value\tabularnewline
\hline 
\hline 
Tango::OPERATOR & 0\tabularnewline
\hline 
Tango::EXPERT & 1\tabularnewline
\hline 
\end{longtable}
\par\end{center}

\vspace{0.3cm}

This parameter allows a graphical application to support two types
of operation :
\begin{itemize}
\item An operator mode for day to day operation
\item An expert mode when tuning is necessary
\end{itemize}
According to this parameter, a graphical application knows if the
attribute is for the operator mode or for the expert mode.

\subsubsection{The root attribute\index{root-attribute} name parameter}

In case the attribute is a forwarded one, this parameter is the name
of the associated root attribute. In case of classical attribute,
this string is set to \textquotedbl{}Not specified\textquotedbl{}.

\subsection{Modifiable attribute parameters}

Each attribute has a configuration set of 20 modifiable parameters.
These can be grouped in three different purposes:
\begin{enumerate}
\item General purpose parameters
\item Alarm related parameters
\item Event related parameters
\end{enumerate}

\subsubsection{General purpose parameters}

Eight attribute parameters are modifiable at run-time via a device
call or via the property database.

\vspace{0.3cm}

\begin{center}
\begin{longtable}{|c|c|}
\hline 
Parameter name & Parameter description\tabularnewline
\hline 
\hline 
description & Attribute description\tabularnewline
\hline 
label & Attribute label\tabularnewline
\hline 
unit & Attribute unit\tabularnewline
\hline 
standard\_unit & Conversion factor to MKSA unit\tabularnewline
\hline 
display\_unit & The attribute unit in a printable form\tabularnewline
\hline 
format & How to print attribute value\tabularnewline
\hline 
min\_value & Attribute min value\tabularnewline
\hline 
max\_value & Attribute max value\tabularnewline
\hline 
enum\_labels & Enumerated labels\tabularnewline
\hline 
memorized & Attribute memorization\tabularnewline
\hline 
\end{longtable}
\par\end{center}

\vspace{0.3cm}

The \textbf{description\index{description}} parameter describes the
attribute. The \textbf{label\index{label}} parameter is used by graphical
application to display a label when this attribute is used in a graphical
application. The \textbf{unit\index{unit}} parameter is the attribute
value unit. The \textbf{standard\_unit\index{standard-unit}} parameter
is the conversion factor to get attribute value in MKSA units. Even
if this parameter is a number, it is returned as a string by the device
\emph{get\_attribute\_config} call. The \textbf{display\_unit\index{display-unit}}
parameter is the string used by graphical application to display attribute
unit to application user. The \textbf{enum\_labels\index{enum_labels@enum\_labels}}
parameter is defined only for attribute of the DEV\_ENUM data type.
This is a vector of strings with one string for each enumeration label.
It is an ordered list.

\paragraph{The format\index{format} attribute parameter}

This parameter specifies how the attribute value should be printed.
It is not valid for string attribute. This format is a string of C++
streams manipulators separated by the \textbf{;} character. The supported
manipulators are :
\begin{itemize}
\item fixed
\item scientific
\item uppercase
\item showpoint
\item showpos
\item setprecision()
\item setw()
\end{itemize}
Their definition are the same than for C++ streams. An example of
format parameter is \begin{center}scientific;uppercase;setprecision(3)\end{center}.
A class called Tango::AttrManip has been written to handle this format
string. Once the attribute format string has been retrieved from the
device, its value can be printed with \begin{center}cout <\textcompwordmark{}<
Tango::AttrManip(format) <\textcompwordmark{}< value <\textcompwordmark{}<
endl;\end{center}.

\paragraph{The min\_value\index{min-value} and max\_value\index{max-value}
parameters}

These two parameters have a meaning only for attribute of the Tango::WRITE
read/write type and for numerical data types. Trying to set the value
of an attribute to something less than or equal to the min\_value
parameter is an error. Trying to set the value of the attribute to
something more or equal to the max\_value parameter is also an error.
Even if these parameters are numbers, they are returned as strings
by the device \emph{get\_attribute\_config()} call. 

These two parameters have no meaning for attribute with data type
DevString, DevBoolean or DevState. An exception is thrown in case
the user try to set them for attribute of these 3 data types.

\paragraph{The memorized\index{memorized} attribute parameter}

This parameter describes the attribute memorization. It is an enumeration
with the following values:
\begin{itemize}
\item NOT\_KNOWN : The device is too old to return this information.
\item NONE : The attribute is not memorized
\item MEMORIZED : The attribute is memorized
\item MEMORIZED\_WRITE\_INIT : The attribute is memorized and the memorized
value is applied at device initialization time.
\end{itemize}

\subsubsection{The alarm related configuration parameters}

Six alarm related attribute parameters are modifiable at run-time
via a device call or via the property database. 

\vspace{0.3cm}

\begin{center}
\begin{longtable}{|c|c|}
\hline 
Parameter name & Parameter description\tabularnewline
\hline 
\hline 
min\_alarm & Attribute low level alarm\tabularnewline
\hline 
max\_alarm & Attribute high level alarm\tabularnewline
\hline 
min\_warning & Attribute low level warning\tabularnewline
\hline 
max\_warning & Attribute high level warning\tabularnewline
\hline 
delta\_t & delta time for RDS alarm (mS)\tabularnewline
\hline 
delta\_val & delta value for RDS alarm (absolute)\tabularnewline
\hline 
\end{longtable}
\par\end{center}

\vspace{0.3cm}
These parameters have no meaning for attribute with data type DevString,
DevBoolean or DevState. An exception is thrown in case the user try
to set them for attribute of these 3 data types.

\paragraph{The min\_alarm\index{min-alarm} and max\_alarm\index{max-alarm}
parameters}

These two parameters have a meaning only for attribute of the Tango::READ,
Tango::READ\_WRITE and Tango::READ\_WITH\_WRITE read/write type and
for numerical data type. When the attribute is read, if its value
is something less than or equal to the min\_alarm parameter or if
it is something more or equal to the max\_alarm parameter, the attribute
quality factor will be set to Tango::ATTR\_ALARM\index{ATTR-ALARM}
and if the device state is Tango::ON, it is switched to Tango::ALARM\index{ALARM}.
Even if these parameters are numbers, they are returned as strings
by the device \emph{get\_attribute\_config()} call.

\paragraph{The min\_warning\index{min-warning} and max\_warning\index{max-warning}
parameters}

These two parameters have a meaning only for attribute of the Tango::READ,
Tango::READ\_WRITE and Tango::READ\_WITH\_WRITE read/write type and
for numerical data type. When the attribute is read, if its value
is something less than or equal to the min\_warning parameter or if
it is something more or equal to the max\_warning parameter, the attribute
quality factor will be set to Tango::ATTR\_WARNING\index{ATTR-WARNING}
and if the device state is Tango::ON, it is switched to Tango::ALARM\index{ALARM}.
Even if these parameters are numbers, they are returned as strings
by the device \emph{get\_attribute\_config()} call.

\paragraph{The delta\_t\index{delta-t} and delta\_val\index{delta-val} parameters}

These two parameters have a meaning only for attribute of the Tango::READ\_WRITE
and Tango::READ\_WITH\_WRITE read/write type and for numerical data
type. They specify if and how the RDS\index{RDS} alarm is used. When
the attribute is read, if the difference between its read value and
the last written value is something more than or equal to the delta\_val
parameter and if at least delta\_val milli seconds occurs since the
last write operation, the attribute quality factor will be set to
Tango::ATTR\_ALARM\index{ATTR-ALARM} and if the device state is Tango::ON,
it is switched to Tango::ALARM\index{ALARM}. Even if these parameters
are numbers, they are returned as strings by the device \emph{get\_attribute\_config()}
call.

\subsubsection{The event related configuration parameters}

Six event\index{event} related attribute parameters are modifiable
at run-time via a device call or via the property database.

\vspace{0.3cm}

\begin{center}
\begin{longtable}{|c|c|}
\hline 
Parameter name & Parameter description\tabularnewline
\hline 
\hline 
rel\_change & Relative change triggering change event\tabularnewline
\hline 
abs\_change & Absolute change triggering change event\tabularnewline
\hline 
\hline 
period  & Period for periodic event\tabularnewline
\hline 
\hline 
archive\_rel\_change  & Relative change for archive event\tabularnewline
\hline 
archive\_abs\_change & Absolute change for archive event\tabularnewline
\hline 
archive\_period  & Period for change archive event\tabularnewline
\hline 
\end{longtable}
\par\end{center}

\vspace{0.3cm}


\paragraph{The rel\_change and abs\_change parameters}

Rel\_change\index{rel-change} is a property with a maximum of 2 values
(comma separated). It specifies the increasing and decreasing relative
change of the attribute value (w.r.t. the value of the previous change
event) which will trigger the event. If the attribute is a spectrum
or an image then a change event is generated if any one of the attribute
value's satisfies the above criterium. It's the absolute value of
these values which is taken into account. If only one value is specified
then it is used for the increasing and decreasing change.

Abs\_change\index{abs-change} is a property of maximum 2 values (comma
separated). It specifies the increasing and decreasing absolute change
of the attribute value (w.r.t the value of the previous change event)
which will trigger the event. If the attribute is a spectrum or an
image then a change event is generated if any one of the attribute
value's satisfies the above criterium. If only one value is specified
then it is used for the increasing and decreasing change. If no values
are specified then the relative change is used.

\paragraph{The periodic period\index{period} parameter}

The minimum time between events (in milliseconds). If no property
is specified then a default value of 1 second is used.

\paragraph{The archive\_rel\_change\index{archive-rel-change}, archive\_abs\_change\index{archive-abs-change}
and archive\_period\index{archive-period} parameters}

archive\_rel\_change is an array property of maximum 2 values which
specifies the positive and negative relative change w.r.t. the previous
attribute value which will trigger the event. If the attribute is
a spectrum or an image then an archive event is generated if any one
of the attribute value's satisfies the above criterium. If only one
property is specified then it is used for the positive and negative
change. If no properties are specified then a default fo +-10\% is
used

archive\_abs\_change is an array property of maximum 2 values which
specifies the positive and negative absolute change w.r.t the previous
attribute value which will trigger the event. If the attribute is
a spectrum or an image then an archive event is generated if any one
of the attribute value's satisfies the above criterium. If only one
property is specified then it is used for the positive and negative
change. If no properties are specified then the relative change is
used.

archive\_period is the minimum time between archive events (in milliseconds).
If no property is specified, no periodic archiving events are send.

\subsection{Setting modifiable attribute parameters}

A default value is given to all modifiable attribute parameters by
the Tango core classes. Nevertheless, it is possible to modify these
values in source code at attribute creation time or via the database.
Values retrieved from the database have a higher priority than values
given at attribute creation time. The attribute parameters are therefore
initialized from:
\begin{enumerate}
\item The Database
\item If nothing in database, from the Tango class default
\item If nothing in database nor in Tango class default, from the library
default value
\end{enumerate}
The default value set by the Tango core library are

\vspace{0.3cm}

\begin{center}
\begin{longtable}{|c|c|c|}
\hline 
Parameter type & Parameter name & Library default value\tabularnewline
\hline 
\hline 
 & description & \textquotedbl{}No description\textquotedbl{}\tabularnewline
\cline{2-3} 
\multicolumn{1}{|c|}{} & label & attribute name\tabularnewline
\cline{2-3} 
\multicolumn{1}{|c|}{} & unit & One empty string\tabularnewline
\cline{2-3} 
\multicolumn{1}{|c|}{general} & standard\_unit & \textquotedbl{}No standard unit\textquotedbl{}\tabularnewline
\cline{2-3} 
\multicolumn{1}{|c|}{purpose} & display\_unit & \textquotedbl{}No display unit\textquotedbl{}\tabularnewline
\cline{2-3} 
\multicolumn{1}{|c|}{} & format & 6 characters with 2 decimal\tabularnewline
\cline{2-3} 
\multicolumn{1}{|c|}{} & min\_value & \textquotedbl{}Not specified\textquotedbl{}\tabularnewline
\cline{2-3} 
\multicolumn{1}{|c|}{} & max\_value & \textquotedbl{}Not specified\textquotedbl{}\tabularnewline
\hline 
\multicolumn{1}{|c|}{} & min\_alarm & \textquotedbl{}Not specified\textquotedbl{}\tabularnewline
\cline{2-3} 
\multicolumn{1}{|c|}{} & max\_alarm & \textquotedbl{}Not specified\textquotedbl{}\tabularnewline
\cline{2-3} 
\multicolumn{1}{|c|}{alarm } & min\_warning & \textquotedbl{}Not specified\textquotedbl{}\tabularnewline
\cline{2-3} 
\multicolumn{1}{|c|}{parameters} & max\_warning & \textquotedbl{}Not specified\textquotedbl{}\tabularnewline
\cline{2-3} 
\multicolumn{1}{|c|}{} & delta\_t & \textquotedbl{}Not specified\textquotedbl{}\tabularnewline
\cline{2-3} 
\multicolumn{1}{|c|}{} & delta\_val & \textquotedbl{}Not specified\textquotedbl{}\tabularnewline
\hline 
\multicolumn{1}{|c|}{} & rel\_change & \textquotedbl{}Not specified\textquotedbl{}\tabularnewline
\cline{2-3} 
\multicolumn{1}{|c|}{} & abs\_change & \textquotedbl{}Not specified\textquotedbl{}\tabularnewline
\cline{2-3} 
\multicolumn{1}{|c|}{event} & period & 1000 (mS)\tabularnewline
\cline{2-3} 
\multicolumn{1}{|c|}{parameters} & archive\_rel\_change & \textquotedbl{}Not specified\textquotedbl{}\tabularnewline
\cline{2-3} 
\multicolumn{1}{|c|}{} & archive\_abs\_change & \textquotedbl{}Not specified\textquotedbl{}\tabularnewline
\cline{2-3} 
\multicolumn{1}{|c|}{} & archive\_period & \textquotedbl{}Not specified\textquotedbl{}\tabularnewline
\hline 
\end{longtable}
\par\end{center}

\vspace{0.3cm}

It is possible to set modifiable parameters via the database at two
levels :
\begin{enumerate}
\item At class level
\item At device level. Each device attribute have all its modifiable parameters
sets to the value defined at class level. If the setting defined at
class level is not correct for one device, it is possible to re-define
it.
\end{enumerate}
If we take the example of a class called \emph{BumperPowerSupply}
with three devices called \emph{sr/bump/1}, \emph{sr/bump/2} and \emph{sr/bump/3}
and one attribute called \emph{wanted\_current}. For the first two
bumpers, the max\_value is equal to 500. For the third one, the max\_value
is only 400. If the max\_value parameter is defined at class level
with the value 500, all devices will have 500 as max\_value for the
\emph{wanted\_current} attribute. It is necessary to re-defined this
parameter at device level in order to have the max\_value for device
sr/bump/3 set to 400.

For the description, label, unit, standard\_unit, display\_unit and
format parameters, it is possible to return them to their default
value by setting them to an empty string.

\subsection{Resetting modifiable attribute parameters}

It is possible to reset attribute parameters to their default value
at any moment. This could be done via the network call available through
the DeviceProxy::set\_attribute\_config() method family. This call
takes attribute parameters as strings. The following table describes
which string has to be used to reset attribute parameters to their
default value. In this table, the user default are the values given
within Pogo in the \textquotedbl{}Properties\textquotedbl{} tab of
the attribute edition window (or in in Tango class code using the
Tango::UserDefaultAttrProp class).\vspace{0.3cm}

\begin{center}
\begin{longtable}{|c|c|}
\hline 
Input string & Action\tabularnewline
\hline 
\hline 
\textquotedbl{}Not specified\textquotedbl{} & Reset to \textbf{library} default\tabularnewline
\hline 
\multirow{2}{*}{\textquotedbl{}\textquotedbl{} (empty string)} & \multirow{2}{*}{Reset to \textbf{user} default if any. Otherwise, reset to \textbf{library}
default}\tabularnewline
 & \tabularnewline
\hline 
\multirow{2}{*}{\textquotedbl{}NaN\textquotedbl{}} & Reset to Tango \textbf{class} default if any\tabularnewline
 & Otherwise, reset to \textbf{user} default (if any) or to \textbf{library}
default\tabularnewline
\hline 
\end{longtable}
\par\end{center}

\vspace{0.3cm}

\begin{center}
Let's take one exemple: For one attribute belonging to a device, we
have the following attribute parameters:\vspace{0.3cm}
\begin{longtable}{|c|c|c|c|}
\hline 
Parameter name & Def. class & Def. user & Def. lib\tabularnewline
\hline 
\hline 
standard\_unit &  &  & No standard unit\tabularnewline
\hline 
min\_value &  & 5 & Not specified\tabularnewline
\hline 
max\_value & 50 &  & Not specified\tabularnewline
\hline 
rel\_change & 5 & 10 & Not specified\tabularnewline
\hline 
\end{longtable}
\par\end{center}

\vspace{0.3cm}

The string \textquotedbl{}Not specified\textquotedbl{} sent to each
attribute parameter will set attribute parameter value to \textquotedbl{}No
standard unit\textquotedbl{} for standard\_unit, \textquotedbl{}Not
specified\textquotedbl{} for min\_value, \textquotedbl{}Not specified\textquotedbl{}
for max\_value and \textquotedbl{}Not specified\textquotedbl{} as
well for rel\_change. The empty string sent to each attribute parameter
will result with \textquotedbl{}No stanadard unit\textquotedbl{} for
standard\_unit, 5 for min\_value, \textquotedbl{}Not specified\textquotedbl{}
for max\_value and 10 for rel\_change. The string \textquotedbl{}NaN\textquotedbl{}
will give \textquotedbl{}No standard unit\textquotedbl{} for standard\_unit,
5 for min\_value, 50 for max\_value and 5 for rel\_change.

C++ specific: Instead of the string \textquotedbl{}Not specified\textquotedbl{}
and \textquotedbl{}NaN\textquotedbl{}, the preprocessor define\textbf{
AlrmValueNotSpec} and \textbf{NotANumber} can be used.

\section{Device pipe}

Pipe are configured with two kind of parameters: Parameters hard-coded
in source code and modifiable parameters

\subsection{Hard-coded device pipe parameters}

Three pipe parameters are defined at pipe\index{pipe} creation time
in the Tango class source code. Obviously, these parameters are not
modifiable except with a new source code compilation. These parameters
are 

\vspace{0.3cm}

\begin{center}
\begin{longtable}{|c|c|}
\hline 
Parameter name & Parameter description\tabularnewline
\hline 
\hline 
name & Pipe name\tabularnewline
\hline 
writable\index{writable} & Pipe read/write type\tabularnewline
\hline 
disp\_level\index{disp_level@disp\_level} & Pipe display level\tabularnewline
\hline 
\end{longtable}
\par\end{center}

\subsubsection{The pipe read/write type. }

Tango supports two kinds of read/write pipe which are :
\begin{itemize}
\item Tango::PIPE\_READ\index{READ} for read only pipe
\item Tango::PIPE\_READ\_WRITE\index{READ-WRITE} for pipe which can be
read and written
\end{itemize}

\subsubsection{The pipe display level\index{level} parameter}

This parameter is only an help for graphical application. It is a
C++ enumeration starting at 0. The code associated with each pipe
display level is defined in the following table (Tango::DispLevel\index{DispLevel}).

\vspace{0.3cm}

\begin{center}
\begin{longtable}{|c|c|}
\hline 
name & Value\tabularnewline
\hline 
\hline 
Tango::OPERATOR & 0\tabularnewline
\hline 
Tango::EXPERT & 1\tabularnewline
\hline 
\end{longtable}
\par\end{center}

\vspace{0.3cm}

This parameter allows a graphical application to support two types
of operation :
\begin{itemize}
\item An operator mode for day to day operation
\item An expert mode when tuning is necessary
\end{itemize}
According to this parameter, a graphical application knows if the
pipe is for the operator mode or for the expert mode.

\subsection{Modifiable pipe parameters}

Each pipe has a configuration set of 2 modifiable parameters. These
parameters are modifiable at run-time via a device call or via the
property database.

\vspace{0.3cm}

\begin{center}
\begin{longtable}{|c|c|}
\hline 
Parameter name & Parameter description\tabularnewline
\hline 
\hline 
description & Pipe description\tabularnewline
\hline 
label & Pipe label\tabularnewline
\hline 
\end{longtable}
\par\end{center}

\vspace{0.3cm}

The \textbf{description\index{description}} parameter describes the
pipe. The \textbf{label\index{label}} parameter is used by graphical
application to display a label when this pipe is used in a graphical
application.

\subsection{Setting modifiable pipe parameters}

A default value is given to all modifiable pipe parameters by the
Tango core classes. Nevertheless, it is possible to modify these values
in source code at pipe creation time or via the database. Values retrieved
from the database have a higher priority than values given at pipe
creation time. The pipe parameters are therefore initialized from:
\begin{enumerate}
\item The Database
\item If nothing in database, from the Tango class default
\item If nothing in database nor in Tango class default, from the library
default value
\end{enumerate}
The default value set by the Tango core library are

\vspace{0.3cm}

\begin{center}
\begin{longtable}{|c|c|}
\hline 
Parameter name & Library default value\tabularnewline
\hline 
\hline 
description & \textquotedbl{}No description\textquotedbl{}\tabularnewline
\hline 
label & pipe name\tabularnewline
\hline 
\end{longtable}
\par\end{center}

\vspace{0.3cm}

It is possible to set modifiable parameters via the database at two
levels :
\begin{enumerate}
\item At class level
\item At device level. Each device pipe have all its modifiable parameters
sets to the value defined at class level. If the setting defined at
class level is not correct for one device, it is possible to re-define
it.
\end{enumerate}
This is the same principle than the one used for attribute configuration
modifiable parameters.

\subsection{Resetting modifiable pipe parameters}

It is possible to reset pipe parameters to their default value at
any moment. This could be done via the network call available through
the DeviceProxy::set\_pipe\_config() method family. It uses the same
principle than the one used for resetting modifiable attribute pipe
parameters. Refer to their documentation if you want to know details
about this feature.

\section{Device class parameter}

A device documentation\index{documentation} field is also defined
at Tango device class level. It is defined as Tango device class level
because each device belonging to a Tango device class should have
the same behaviour and therefore the same documentation. This field
is store in the DeviceClass class. It is possible to set this field
via a class property. This property name is \begin{center}class name->doc\_url\end{center} and
is retrieved when instance of the DeviceClass object is created. A
default value is defined for this field.

\section{The device black box}

This black box\index{black-box} is a help tool to ease debugging
session for a running device server. The TANGO core software records
every device request in this black box. A tango client is able to
retrieve the black box contents with a specific CORBA\index{CORBA}
operation availabble for every device. Each black box entry is returned
as a string with the following information :
\begin{itemize}
\item The date where the request has been executed by the device. The date
format is dd/mm/yyyy hh24:mi:ss:SS (The last field is the second hundredth
number).
\item The type of CORBA requests. In case of attributes, the name of the
requested attribute is returned. In case of operation, the operation
type is returned. For ``command\_inout'' operation, the command
name is returned.
\item The client host name
\end{itemize}

\section{Automatically added commands}

As already mentionned in this documentation, each Tango device supports
at least three commands which are State\index{State}, Status\index{Status}
and Init\index{Init}. The following array details command input and
output data type

\vspace{0.3cm}

\begin{center}
\begin{longtable}{|c|c|c|}
\hline 
Command name & Input data type & Output data type\tabularnewline
\hline 
\hline 
State & void & Tango::DevState\tabularnewline
\hline 
Status & void & Tango::DevString\tabularnewline
\hline 
Init & void & void\tabularnewline
\hline 
\end{longtable}
\par\end{center}

\vspace{0.3cm}


\subsection{The State command}

This command gets the device state (stored in its \emph{device\_state}
data member) and returns it to the caller. The device state is a variable
of the Tango\_DevState type (packed into a CORBA Any object when it
is returned by a command)

\subsection{The Status command}

This command gets the device status (stored in its \emph{device\_status}
data member) and returns it to the caller. The device status is a
variable of the string type.

\subsection{The Init command}

This commands re-initialise a device keeping the same network connection.
After an Init command executed on a device, it is not necessary for
client to re-connect to the device. This command first calls the device
\emph{delete\_device()} method and then execute its \emph{init\_device()}
method. For C++ device server, all the memory allocated in the \emph{init\_device()}
method must be freed in the \emph{delete\_device()} method. The language
device desctructor automatically calls the \emph{delete\_device()}
method.

\section{DServer\index{DServer} class device commands}

As already explained in \ref{DServer_class}, each device server process
has its own Tango device. This device supports the three commands
previously described plus 32 commands which are DevRestart\index{DevRestart},
RestartServer\index{RestartServer}, QueryClass\index{QueryClass},
QueryDevice\index{QueryDevice}, Kill\index{Kill}, QueryWizardClassProperty\index{QueyWizardClassProperty},
QueryWizardDevProperty\index{QueryWizardDevProperty}, QuerySubDevice\index{QuerySubDevice},
the polling related commands which are StartPolling\index{StartPolling},
StopPolling\index{StopPolling}, AddObjPolling\index{AddObjPolling},
RemObjPolling\index{RemObjPolling}, UpdObjPollingPeriod\index{UpdObjPollingPeriod},
PolledDevice\index{PolledDevice} and DevPollStatus\index{DevPollStatus},
the device locking related commands which are LockDevice\index{LockDevice},
UnLockDevice\index{UnLockDevice}, ReLockDevices\index{ReLockDevices}
and DevLockStatus\index{DevLockStatus}, the event related commands
called EventSubscriptionChange\index{EventSubscriptionChange}, ZmqEventSubscriptionChange\index{ZmqEventSubscriptionChange}
and EventConfirmSubscription\index{EventConfirmSubscription} and
finally the logging related commands which are AddLoggingTarget\index{AddLoggingTarget},
RemoveLoggingTarget\index{RemoveLoggingTarget}, GetLoggingTarget\index{GetLoggingTarget},
GetLoggingLevel\index{GetLoggingLevel}, SetLoggingLevel\index{SetLoggingLevel},
StopLogging\index{StopLogging} and StartLogging\index{StartLogging}.
The following table give all commands input and output data types

\vspace{0.3cm}

\begin{center}
\begin{longtable}{|c|c|c|}
\hline 
Command name & Input data type  & Output data type\tabularnewline
\hline 
\hline 
State & void & Tango::DevState\tabularnewline
\hline 
Status & void & Tango::DevString\tabularnewline
\hline 
Init & void & void\tabularnewline
\hline 
DevRestart & Tango::DevString & void\tabularnewline
\hline 
RestartServer & void & void\tabularnewline
\hline 
QueryClass & void & Tango::DevVarStringArray\tabularnewline
\hline 
QueryDevice & void & Tango::DevVarStringArray\tabularnewline
\hline 
Kill & void & void\tabularnewline
\hline 
QueryWizardClassProperty & Tango::DevString & Tango::DevVarStringArray\tabularnewline
\hline 
QueryWizardDevProperty & Tango::DevString & Tango::DevVarStringArray\tabularnewline
\hline 
QuerySubDevice & void & Tango::DevVarStringArray\tabularnewline
\hline 
\hline 
StartPolling & void & void\tabularnewline
\hline 
StopPolling & void & void\tabularnewline
\hline 
AddObjPolling & Tango::DevVarLongStringArray & void\tabularnewline
\hline 
RemObjPolling & Tango::DevVarStringArray & void\tabularnewline
\hline 
UpdObjPollingPeriod & Tango::DevVarLongStringArray & void\tabularnewline
\hline 
PolledDevice & void & Tango::DevVarStringArray\tabularnewline
\hline 
DevPollStatus & Tango::DevString & Tango::DevVarStringArray\tabularnewline
\hline 
\hline 
LockDevice & Tango::DevVarLongStringArray & void\tabularnewline
\hline 
UnLockDevice & Tango::DevVarLongStringArray & Tango::DevLong\tabularnewline
\hline 
ReLockDevices & Tango::DevVarStringArray & void\tabularnewline
\hline 
DevLockStatus & Tango::DevString & Tango::DevVarLongStringArray\tabularnewline
\hline 
\hline 
EventSubscribeChange & Tango::DevVarStringArray & Tango::DevLong\tabularnewline
\hline 
ZmqEventSubscriptionChange & Tango::DevVarStringArray & Tango::DevVarLongStringArray\tabularnewline
\hline 
EventConfirmSubscription & Tango::DevVarStringArray & void\tabularnewline
\hline 
\hline 
AddLoggingTarget  & Tango::DevVarStringArray & void\tabularnewline
\hline 
RemoveLoggingTarget  & Tango::DevVarStringArray & void\tabularnewline
\hline 
GetLoggingTarget  & Tango::DevString & Tango::DevVarStringArray\tabularnewline
\hline 
GetLoggingLevel & Tango::DevVarStringArray & Tango::DevVarLongStringArray\tabularnewline
\hline 
SetLoggingLevel & Tango::DevVarLongStringArray & void\tabularnewline
\hline 
StopLogging & void & void\tabularnewline
\hline 
StartLogging & void & void\tabularnewline
\hline 
\end{longtable}
\par\end{center}

\vspace{0.3cm}

The device description field is set to ``A device server device''.
Device server started with the -file command line option also supports
a command called QueryEventChannelIOR\index{QueryEventChannelIOR}.
This command is used interanally by the Tango kernel classes when
the event system is used with device server using database on file.

\subsection{The State command}

This device state is always set to ON

\subsection{The Status command}

This device status is always set to ``The device is ON'' followed
by a new line character and a string describing polling thread status.
This string is either ``The polling is OFF'' or ``The polling is
ON'' according to polling state.

\subsection{The DevRestart command}

The DevRestart command restart a device. The name of the device to
be re-started is the command input parameter. The command destroys
the device by calling its destructor and re-create it from its constructor.

\subsection{The RestartServer command}

The DevRestartServer command restarts all the device pattern(s) embedded
in the device server process. Therefore, all the devices implemented
in the server process are destroyed and re-built\footnote{Their black-box is also destroyed and re-built}.
The network connection between client(s) and device(s) implemented
in the device server process is destroyed and re-built. 

Executing this command allows a complete restart of the device server
without stopping the process.

\subsection{The QueryClass\index{QueryClass} command}

This command returns to the client the list of Tango device class(es)
embedded in the device server. It returns only class(es) implemented
by the device server programmer. The DServer device class name (implemented
by the TANGO core software) is not returned by this command.

\subsection{The QueryDevice\index{QueryDevice} command}

This command returns to the client the list of device name for all
the device(s) implemented in the device server process. Each device
name is returned using the following syntax : \begin{center}<class
name>::<device name>\end{center}

The name of the DServer class device is not returned by this command.

\subsection{The Kill\index{Kill} command}

This command stops the device server process. In order that the client
receives a last answer from the server, this command starts a thread
which will after a short delay, kills the device server process.

\subsection{The QueryWizardClassProperty\index{QueryWizardClassProperty} command}

This command returns the list of property(ies) defined for a class
stored in the device server process property wizard. For each property,
its name, a description and a default value is returned.

\subsection{The QueryWizardDevProperty\index{QueryWizardDevProperty} command}

This command returns the list of property(ies) defined for a device
stored in the device server process property wizard. For each property,
its name, a description and a default value is returned.

\subsection{The QuerySubDevice\index{QuerySubDevice} command}

This command returns the list of sub-device(s) imported by each device
within the server. A sub-device is a device used ( to execute command(s)
and/or to read/write attribute(s) ) by one of the device server process
devices. There is one element in the returned strings array for each
sub-device. The syntax of each string is the device name, a space
and the sub-device name. In case of device server process starting
threads using a sub-device, it is not possible to link this sub-device
to any process devices. In such a case, the string contains only the
sub-device name

\subsection{The StartPolling\index{StartPolling} command}

This command starts the polling thread

\subsection{The StopPolling\index{StopPolling} command}

This command stops the polling thread

\subsection{The AddObjPolling\index{AddObjPolling} command}

This command adds a new object in the list of object(s) to be polled.
The command input parameters are embedded within a Tango::DevVarLongStringArray
data type with one long data and three strings. The input parameters
are:

\vspace{0.3cm}

\begin{center}
\begin{longtable}{|c|c|}
\hline 
Command parameter & Parameter meaning\tabularnewline
\hline 
\hline 
svalue{[}0{]} & Device name\tabularnewline
\hline 
svalue{[}1{]} & Object type (``command`` or ``attribute``)\tabularnewline
\hline 
svalue{[}2{]} & Object name\tabularnewline
\hline 
lvalue{[}0{]} & polling period in mS\tabularnewline
\hline 
\end{longtable}
\par\end{center}

\vspace{0.3cm}

The object type string is case independent. The object name string
(command name or attribute name) is case dependant. This command does
not start polling if it is stopped. This command is not allowed in
case the device is locked and the command requester is not the lock
owner.

\subsection{The RemObjPolling\index{RemObjPolling} command}

This command removes an object of the list of polled objects. The
command input data type is a Tango::DevVarStringArray with three strings.
These strings meaning are :

\vspace{0.3cm}

\begin{center}
\begin{longtable}{|c|c|}
\hline 
String & Meaning\tabularnewline
\hline 
\hline 
string{[}0{]} & Device name\tabularnewline
\hline 
string{[}1{]} & Object type (``command`` or ``attribute``)\tabularnewline
\hline 
string{[}2{]} & Object name\tabularnewline
\hline 
\end{longtable}
\par\end{center}

\vspace{0.3cm}

The object type string is case independent. The object name string
(command name or attribute name) is case dependant. This command is
not allowed in case the device is locked and the command requester
is not the lock owner.

\subsection{The UpdObjPollingPeriod\index{UpdObjPollingPeriod} command}

This command changes the polling period for a specified object. The
command input parameters are embedded within a Tango::DevVarLongStringArray
data type with one long data and three strings. The input parameters
are:

\vspace{0.3cm}

\begin{center}
\begin{longtable}{|c|c|}
\hline 
Command parameter & Parameter meaning\tabularnewline
\hline 
\hline 
svalue{[}0{]} & Device name\tabularnewline
\hline 
svalue{[}1{]} & Object type (``command`` or ``attribute``)\tabularnewline
\hline 
svalue{[}2{]} & Object name\tabularnewline
\hline 
lvalue{[}0{]} & new polling period in mS\tabularnewline
\hline 
\end{longtable}
\par\end{center}

\vspace{0.3cm}

The object type string is case independent. The object name string
(command name or attribute name) is case dependant. This command does
not start polling if it is stopped. This command is not allowed in
case the device is locked and the command requester is not the lock
owner.

\subsection{The PolledDevice\index{PolledDevice} command}

This command returns the name of device which are polled. Each string
in the Tango::DevVarStringArray returned by the command is a device
name which has at least one command or attribute polled. The list
is alphabetically sorted.

\subsection{The DevPollStatus\index{DevPollStatus} command}

This command returns a polling status for a specific device. The input
parameter is a device name. Each string in the Tango::DevVarStringArray
returned by the command is the polling status for each polled device
objects (command or attribute). For each polled objects, the polling
status is :
\begin{itemize}
\item The object name
\item The object polling period (in mS)
\item The object polling ring buffer depth
\item The time needed (in mS) for the last command execution or attribute
reading
\item The time since data in the ring buffer has not been updated. This
allows a check of the polling thread
\item The delta time between the last records in the ring buffer. This allows
checking that the polling period is respected by the polling thread.
\item The exception parameters in case of the last command execution or
the last attribute reading failed.
\end{itemize}
A new line character is inserted between each piece of information.

\subsection{The LockDevice\index{LockDevice} command}

This command locks a device for the calling process. The command input
parameters are embedded within a Tango::DevVarLongStringArray data
type with one long data and one string. The input parameters are:\vspace{0.3cm}

\begin{center}
\begin{longtable}{|c|c|}
\hline 
Command parameter & Parameter meaning\tabularnewline
\hline 
\hline 
svalue{[}0{]} & Device name\tabularnewline
\hline 
lvalue{[}0{]} & Lock validity\tabularnewline
\hline 
\end{longtable}
\par\end{center}

\vspace{0.3cm}


\subsection{The UnLockDevice\index{UnLockDevice} command}

This command unlocks a device. The command input parameters are embedded
within a Tango::DevVarLongStringArray data type with one long data
and one string. The input parameters are:\vspace{0.3cm}

\begin{center}
\begin{longtable}{|c|c|}
\hline 
Command parameter & Parameter meaning\tabularnewline
\hline 
\hline 
svalue{[}0{]} & Device name\tabularnewline
\hline 
lvalue{[}0{]} & Force flag\tabularnewline
\hline 
\end{longtable}
\par\end{center}

\vspace{0.3cm}

The force flag parameter allows a client to unlock a device already
locked by another process (for admin usage only)

\subsection{The ReLockDevices\index{ReLockDevices} command}

This command re-lock devices. The input argument is the list of devices
to be re-locked. It's an error to re-lock a device which is not already
locked.

\subsection{The DevLockStatus\index{DevLockStatus} command}

This command returns a device locking status to the caller. Its input
parameter is the device name. The output parameters are embedded within
a Tango::DevVarLongStringArray data type with three strings and six
long. These data are\vspace{0.3cm}

\begin{center}
\begin{longtable}{|c|c|}
\hline 
Command parameter & Parameter meaning\tabularnewline
\hline 
\hline 
svalue{[}0{]} & Locking string\tabularnewline
\hline 
svalue{[}1{]} & CPP client host IP address or \textquotedbl{}Not defined\textquotedbl{}\tabularnewline
\hline 
svalue{[}2{]} & Java VM main class for Java client or \textquotedbl{}Not defined\textquotedbl{}\tabularnewline
\hline 
lvalue{[}0{]} & Lock flag (1 if locked, 0 othterwise)\tabularnewline
\hline 
lvalue{[}1{]} & CPP client host IP address or 0 for Java locker\tabularnewline
\hline 
lvalue{[}2{]} & Java locker UUID part 1or 0 for CPP locker\tabularnewline
\hline 
lvalue{[}3{]} & Java locker UUID part 2 or 0 for CPP locker\tabularnewline
\hline 
lvalue{[}4{]} & Java locker UUID part 3 or 0 for CPP locker\tabularnewline
\hline 
lvalue{[}5{]} & Java locker UUID part 4 or 0 for CPP locker\tabularnewline
\hline 
\end{longtable}
\par\end{center}

\vspace{0.3cm}


\subsection{The EventSubscriptionChange\index{EventSubscriptionChange} command
(C++ server only)}

This command is used as a piece of the \textquotedbl{}heartbeat\textquotedbl{}
system between an event client and the device server generating the
event. There is no reason to generate events if there is no client
which has subscribed to it. It is used by the \emph{DeviceProxy::subscribe\_event()}
method and one of the event thread on the client side to inform the
server to keep on generating events for the attribute in question.
It reloads the subscription timer with the current time. Events are
not generated when there are no clients subscribed within the last
10 minutes. The input parameters are:

\vspace{0.3cm}

\begin{center}
\begin{longtable}{|c|c|}
\hline 
Command parameter & Parameter meaning\tabularnewline
\hline 
\hline 
argin{[}0{]} & Device name\tabularnewline
\hline 
argin{[}1{]} & Attribute name\tabularnewline
\hline 
argin{[}2{]} & action (\textquotedbl{}subscribe\textquotedbl{} or \textquotedbl{}unsubsribe\textquotedbl{})\tabularnewline
\hline 
argin{[}3{]} & event name (\textquotedbl{}change\textquotedbl{}, \textquotedbl{}periodic\textquotedbl{},
\textquotedbl{}archive\textquotedbl{},\textquotedbl{}attr\_conf\textquotedbl{})\tabularnewline
\hline 
\end{longtable}
\par\end{center}

\vspace{0.3cm}

The command output data is the simply the Tango release used by the
device server process. This is necessary for compatibility reason.

\subsection{The ZmqEventSubscriptionChange\index{ZmqEventSubscriptionChange}
command }

This command is used as a piece of the \textquotedbl{}heartbeat\textquotedbl{}
system between an event client and the device server generating the
event when client and/or device server uses Tango release 8 or above.
There is no reason to generate events if there is no client which
has subscribed to it. It is used by the \emph{DeviceProxy::subscribe\_event()}
method and one of the event thread on the client side to inform the
server to keep on generating events for the attribute in question.
It reloads the subscription timer with the current time. Events are
not generated when there are no clients subscribed within the last
10 minutes. The input parameters are the same than the one used for
the EventSubscriptionChange command. They are:

\vspace{0.3cm}

\begin{center}
\begin{longtable}{|c|c|}
\hline 
Command in parameter & Parameter meaning\tabularnewline
\hline 
\hline 
argin{[}0{]} & Device name\tabularnewline
\hline 
argin{[}1{]} & Attribute name\tabularnewline
\hline 
argin{[}2{]} & action (\textquotedbl{}subscribe\textquotedbl{} or \textquotedbl{}unsubsribe\textquotedbl{})\tabularnewline
\hline 
argin{[}3{]} & event name (\textquotedbl{}change\textquotedbl{}, \textquotedbl{}periodic\textquotedbl{},
\textquotedbl{}archive\textquotedbl{},\textquotedbl{}attr\_conf\textquotedbl{})\tabularnewline
\hline 
\end{longtable}
\par\end{center}

\vspace{0.3cm}

The command output parameters aer all the necessary data to build
one event connection between a client and the device server process
generating the events. This means:\vspace{0.3cm}

\begin{center}
\begin{longtable}{|c|c|}
\hline 
Command out parameter & Parameter meaning\tabularnewline
\hline 
\hline 
svalue{[}0{]} & Heartbeat ZMQ socket connect end point\tabularnewline
\hline 
svalue{[}1{]} & Event ZMQ socket connect end point\tabularnewline
\hline 
lvalue{[}0{]} & Tango lib release used by device server\tabularnewline
\hline 
lvalue{[}1{]} & Device IDL release\tabularnewline
\hline 
lvalue{[}2{]} & Subscriber HWM\tabularnewline
\hline 
lvalue{[}3{]} & Rate (Multicasting related)\tabularnewline
\hline 
lvalue{[}4{]} & IVL (Multicasting related)\tabularnewline
\hline 
\end{longtable}
\par\end{center}

\vspace{0.3cm}


\subsection{The EventConfirmSubscription\index{EventConfirmSubscription} command}

This command is used by client to regularly notify to device server
process their interest in receiving events. If this command is not
received, after a delay of 600 sec (10 mins), event(s) will not be
sent any more. The input parameters for the EventConfirmSubscription
command must be a multiple of 3. They are 3 parameters for each event
confirmed by this command. Per event, these parameters are:

\vspace{0.3cm}

\begin{center}
\begin{longtable}{|c|c|}
\hline 
Command in parameter & Parameter meaning\tabularnewline
\hline 
\hline 
argin{[}x{]} & Device name\tabularnewline
\hline 
argin{[}x + 1{]} & Attribute name\tabularnewline
\hline 
argin{[}x + 2{]} & Event name\tabularnewline
\hline 
\end{longtable}
\par\end{center}

\subsection{The AddLoggingTarget\index{AddLoggingTarget} command}

This command adds one (or more) logging target(s) to the specified
device(s). The command input parameter is an array of string logically
composed of \{device\_name, target\_type::target\_name\} groups where
the elements have the following semantic: 
\begin{itemize}
\item device\_name is the name of the device which logging behavior is to
be controlled. The wildcard \textquotedbl{}{*}\textquotedbl{} is supported
to apply the modification to all devices encapsulated within the device
server (e.g. to ask all devices to log to the same device target).
\item target\_type::target\_name: target\_type is one of the supported target
types and target\_name, the name of the target. Supported target types
are: \emph{console}, \emph{file} and \emph{device}. For a device target,
target\_name must contain the name of a log consumer device (as defined
in \ref{sec:Tango-log-consumer}). For a file target, target\_name
is the full path to the file to log to. If omitted the device's name
is used to build the file name (domain\_family\_member.log). Finally,
target\_name is ignored in the case of a console target and can be
omitted.
\end{itemize}
This command is not allowed in case the device is locked and the command
requester is not the lock owner.

\subsection{The RemoveLoggingTarget\index{RemoveLoggingTarget} command}

Remove one (or more) logging target(s) from the specified device(s).The
command input parameter is an array of string logically composed of
\{device\_name, target\_type::target\_name\} groups where the elements
have the following semantic:
\begin{itemize}
\item device\_name: the name of the device which logging behavior is to
be controlled. The wildcard \textquotedbl{}{*}\textquotedbl{} is supported
to apply the modification to all devices encapsulated within the device
server (e.g. to ask all devices to stop logging to a given device
target).
\item target\_type::target\_name: target\_type is one of the supported target
types and target\_name, the name of the target. Supported target types
are: \emph{console}, \emph{file} and \emph{device}. For a device target,
target\_name must contain the name of a log consumer device (as defined
in \ref{sec:Tango-log-consumer}). For a file target, target\_name
is the full path to the file to log to. If omitted the device's name
is used to build the file name (domain\_family\_member.log). Finally,
target\_name is ignored in the case of a console target and can be
omitted.
\end{itemize}
The wildcard \textquotedbl{}{*}\textquotedbl{} is supported for target\_name.
For instance, RemoveLoggingTarget ({[}\textquotedbl{}{*}\textquotedbl{},
\textquotedbl{}device::{*}\textquotedbl{}{]}) removes all the device
targets from all the devices running in the device server. This command
is not allowed in case the device is locked and the command requester
is not the lock owner.

\subsection{The GetLoggingTarget\index{GetLoggingTarget} command}

Returns the current target list of the specified device. The command
parameter device\_name is the name of the device which logging target
list is requested. The list is returned as a DevVarStringArray containing
target\_type::target\_name elements.

\subsection{The GetLoggingLevel\index{GetLoggingLevel} command}

Returns the logging level of the specified devices. The command input
parameter device\_list contains the names of the devices which logging
target list is requested. The wildcard \textquotedbl{}{*}\textquotedbl{}
is supported to get the logging level of all the devices running within
the server. The string part of the result contains the name of the
devices and its long part contains the levels. Obviously, result.lvalue{[}i{]}
is the current logging level of the device named result.svalue{[}i{]}.

\subsection{The SetLoggingLevel\index{SetLoggingLevel} command}

Changes the logging level of the specified devices. The string part
of the command input parameter contains the device names while its
long part contains the logging levels. The set of possible values
for levels is: 0=OFF, 1=FATAL, 2=ERROR, 3=WARNING, 4=INFO, 5=DEBUG. 

The wildcard \textquotedbl{}{*}\textquotedbl{} is supported to assign
all devices the same logging level. For instance, SetLoggingLevel
({[}\textquotedbl{}{*}\textquotedbl{}{]} {[}3{]}) set the logging
level of all the devices running within the server to WARNING. This
command is not allowed in case the device is locked and the command
requester is not the lock owner.

\subsection{The StopLogging\index{StopLogging} command}

For all the devices running within the server, StopLogging saves their
current logging level and set their logging level to OFF. 

\subsection{The StartLogging\index{StartLogging} command}

For each device running within the server, StartLogging restores their
logging level to the value stored during a previous StopLogging call.

\section{DServer class device properties}

This device has two properties related to polling threads pool management
plus another one for the choice of polling algorithm. These properties
are described in the following table 

\vspace{0.3cm}

\begin{center}
\begin{longtable}{|c|c|c|}
\hline 
Property name & property rule & default value\tabularnewline
\hline 
\hline 
polling\_threads\_pool\_size & Max number of thread in the polling pool & 1\tabularnewline
\hline 
polling\_threads\_pool\_conf & Polling threads pool configuration & \tabularnewline
\hline 
polling\_before\_9 & Choice of the polling algorithm & false\tabularnewline
\hline 
\end{longtable}
\par\end{center}

\vspace{0.3cm}

The rule of the polling\_threads\_pool\_size\index{polling-threads-pool-size}
is to define the maximun number of thread created for the polling
threads pool size. The rule of the polling\_threads\_pool\_conf\index{polling-threads-pool-conf}
is to define which thread in the pool is in charge of all the polled
object(s) of which device. This property is an array of strings with
one string per used thread in the pool. The content of the string
is simply a device name list with device name splitted by a comma.
Example of polling\_threads\_pool\_conf property for 3 threads used:
\begin{lyxcode}
dserver/<ds~exec~name>/<inst.~name>/polling\_threads\_pool\_conf->~the/dev/01

~~~~~~~~~~~~~~~~~~the/dev/02,the/dev/06

~~~~~~~~~~~~~~~~~~the/dev/03
\end{lyxcode}
Thread number 2 is in charge of 2 devices. Note that there is an entry
in this list only for the used threads in the pool.

The rule of the polling\_before\_9\index{polling_before_9@polling\_before\_9}
property is to select the polling algorithm which was used in Tango
device server process before Tango release 9.

\section{Tango log consumer\index{consumer} \label{sec:Tango-log-consumer}}

\subsection{The available Log Consumer}

One implementation of a log consumer associated to a graphical user
interface is available within Tango. It is a standalone java application
called \textbf{LogViewer} based on the publicly available chainsaw
application from the log4j package. It supports two way of running
which are:
\begin{itemize}
\item The static mode: In this mode, LogViewer is started with a parameter
which is the name of the log consumer device implemented by the application.
All messages sent by devices with a logging target type set to \emph{device}
and with a logging target name set to the same device name than the
device name passed as application parameter will be displayed (if
the logging level allows it).
\item The dynamic mode: In this mode, the name of the log consumer device
implemented by the application is build at application startup and
is dynamic. The user with the help of the graphical interface chooses
device(s) for which he want to see log messages.
\end{itemize}

\subsection{The Log Consumer interface}

A Tango Log Consumer device is nothing but a tango device supporting
the following tango command : \begin{center}void log (Tango::DevVarStringArray
details)\end{center} where details is an array of string carrying
the log details. Its structure is:
\begin{itemize}
\item details{[}0{]} : the timestamp in millisecond since epoch (01.01.1970) 
\item details{[}1{]} : the log level
\item details{[}2{]} : the log source (i.e. device name)
\item details{[}3{]} : the log message
\item details{[}4{]} : the log NDC (contextual info) - Not used but reserved
\item details{[}5{]} : the thread identifier (i.e. the thread from which
the log request comes from)
\end{itemize}
These log details can easily be extended. Any tango device supporting
this command can act as a device target for other devices. 

\section{Control system specific}

It is possible to define a few control system parameters. By control
system, we mean for each set of computers having the same database
device server (the same TANGO\_HOST environment variable)

\subsection{The device class documentation default value}

Each control system may have it's own default device class documentation
value. This is defined via a class property. The property name is
\begin{center}Default->doc\_url\end{center} It's retrieved if the
device class itself does not define any doc\_url property. If the
Default->doc\_url property is also not defined, a hard-coded default
value is provided.

\subsection{The services definition}

The property used to defined control system services is named \textbf{Services\index{Services}}
and belongs to the free object \textbf{CtrlSystem}\index{CtrlSystem}.
This property is an array of strings. Each string defines a service
available within the control system. The syntax of each service definition
is \begin{center}Service name/Instance name:service device name\end{center}

\subsection{Tuning the event system buffers (HWM)\index{HWM}}

Starting with Tango release 8, ZMQ\index{ZMQ} is used for the event
based communication between clients and device server processes. ZMQ
implementation provides asynchronous communication in the sense that
the data to be transmitted is first stored in a buffer and then really
sent on the network by dedicated threads. The size of this buffers
(on client and device server side) is called High Water Mark (HWM)
and is tunable. This is tunable at several level.
\begin{enumerate}
\item The library set a default value of \textbf{1000} for both buffers
(client and device server side)
\item Control system properties used to tune these size are named \textbf{DSEventBufferHwm\index{DsEventBufferHwm}}
(device server side) and \textbf{EventBufferHwm\index{EventBufferHwm@\textbf{EventBufferHwm}}}
(client side). They both belongs to the free object \textbf{CtrlSystem}\index{CtrlSystem}.
Each property is the max number of events storable in these buffer.
\item At client or device server level using the library calls \emph{Util::set\_ds\_event\_buffer\_hwm()}
documented in \cite{Tango-dsclasses-doc} or \emph{ApiUtil::set\_event\_buffer\_hwm()
}documented in \ref{sec:Tango::ApiUtil}
\item Using environment variables TANGO\_DS\_EVENT\_BUFFER\_HWM\index{TANGO-DS-EVENT-BUFFER-HWM}
or TANGO\_EVENT\_BUFFER\_HWM\index{TANGO-EVENT-BUFFER-HWM}
\end{enumerate}

\subsection{Allowing NaN when writing attributes (floating point)}

A property named \textbf{WAttrNaNAllowed\index{WAttrNaNAllowed}}
belonging to the free object \textbf{CtrlSystem\index{CtrlSystem}}
allows a Tango control system administrator to allow or disallow NaN
numbers when writing attributes of the DevFloat or DevDouble data
type. This is a boolean property and by default, it's value is taken
as false (Meaning NaN values are rejected).

\subsection{Tuning multicasting event propagation}

Starting with Tango 8.1, it is possible to transfer event(s) between
devices and clients using a multicast protocol. The properties \textbf{MulticastEvent}\index{MulticastEvent},
\textbf{MulticastRate}\index{MulticastRate}, \textbf{MulticastIvl\index{MulticastIvl}}
and \textbf{MulticastHops\index{MulticastHops}} also belonging to
the free object \textbf{CtrlSystem} allow the user to configure which
events has to be sent using multicasting and with which parameters.
See chapter \textquotedbl{}Advanced features/Using multicast protocol
to transfer events\textquotedbl{} to get details about these properties.

\subsection{Summary of CtrlSystem free object properties}

The following table summarizes properties defined at control system
level and belonging to the free object CtrlSystem\index{CtrlSystem}

\vspace{0.3cm}

\begin{center}
\begin{longtable}{|c|c|c|}
\hline 
Property name & property rule & default value\tabularnewline
\hline 
\hline 
Services & List of defined services & No default\tabularnewline
\hline 
DsEventBufferHwm & DS event buffer high water mark & 1000\tabularnewline
\hline 
EventBufferHwm & Client event buffer high water mark & 1000\tabularnewline
\hline 
WAttrNaNAllowed & Allow NaN when writing attr. & false\tabularnewline
\hline 
MulticastEvent & List of multicasting events & No default\tabularnewline
\hline 
MulticastRate & Rate for multicast event transport & 80\tabularnewline
\hline 
MulticastIvl & Time to keep data for re-transmission & 20\tabularnewline
\hline 
MulticastHops & Max number of eleemnts to cross & 5\tabularnewline
\hline 
\end{longtable}
\par\end{center}

\vspace{0.3cm}


\section{C++ specific}

\subsection{The Tango master include file (tango.h\index{tango.h})}

Tango has a master include file called \begin{center}tango.h\end{center} This
master include file includes the following files :
\begin{itemize}
\item Tango configuration include file : \textbf{tango\_config.h}
\item CORBA include file : \textbf{idl/tango.h}
\item Some network include files for WIN32 : \textbf{winsock2.h} and \textbf{mswsock.h}
\item C++ streams include file :

\begin{itemize}
\item \textbf{iostream}, \textbf{sstream} and \textbf{fstream} 
\end{itemize}
\item Some standard C++ library include files : \textbf{memory, string}
and \textbf{vector}
\item A long list of other Tango include files
\end{itemize}

\subsection{Tango specific pre-processor define}

The tango.h previously described also defined some pre-processor macros
allowing Tango release to be checked at compile time. These macros
are:
\begin{itemize}
\item TANGO\_VERSION\_MAJOR\index{TANGO-VERSION-MAJOR}
\item TANGO\_VERSION\_MINOR\index{TANGO-VERSION-MINOR}
\item TANGO\_VERSION\_PATCH\index{TANGO-VERSION-PATCH}
\end{itemize}
For instance, with Tango release 8.1.2, TANGO\_VERSION\_MAJOR will
be set to 8 while TANGO\_VERSION\_MINOR will be 1 and TANGO\_VERSION\_PATCH
will be 2.

\subsection{Tango specific types}

\subsubsection*{Operating system free type}

Some data type used in the TANGO core software have been defined.
They are described in the following table.

\vspace{0.3cm}

\begin{center}
\begin{longtable}{|c|c|}
\hline 
Type name & C++ name\tabularnewline
\hline 
\hline 
TangoSys\_MemStream & stringstream\tabularnewline
\hline 
TangoSys\_OMemStream & ostringstream\tabularnewline
\hline 
TangoSys\_Pid & int\tabularnewline
\hline 
TangoSys\_Cout & ostream\tabularnewline
\hline 
\end{longtable}
\par\end{center}

\vspace{0.3cm}

These types are defined in the tango\_config.h file

\subsubsection{Template command model related type}

As explained in \ref{Command fact}, command created with the template
command model uses static casting. Many type definition have been
written for these casting.

\vspace{0.3cm}

\begin{center}
\begin{longtable}{|c|c|c|}
\hline 
Class name & Command allowed method (if any) & Command execute method\tabularnewline
\hline 
\hline 
TemplCommand\index{TemplCommand} & Tango::StateMethodPtr & Tango::CmdMethPtr\tabularnewline
\hline 
TemplCommandIn\index{TemplCommandIn} & Tango::StateMethodPtr & Tango::CmdMethPtr\_xxx\tabularnewline
\hline 
TemplCommandOut\index{TemplCommandOut} & Tango::StateMethodPtr & Tango::xxx\_CmdMethPtr\tabularnewline
\hline 
TemplCommandInOut\index{TemplCommandInOut} & Tango::StateMethodPtr & Tango::xxx\_CmdMethPtr\_yyy\tabularnewline
\hline 
\end{longtable}
\par\end{center}

\vspace{0.3cm}

The \textbf{Tango::StateMethPtr} is a pointer to a method of the DeviceImpl
class which returns a boolean and has one parameter which is a reference
to a const CORBA::Any obect. 

The \textbf{Tango::CmdMethPtr} is a pointer to a method of the DeviceImpl
class which returns nothing and needs nothing as parameter.

The \textbf{Tango::CmdMethPtr\_xxx} is a pointer to a method of the
DeviceImpl class which returns nothing and has one parameter. xxx
must be set according to the method parameter type as described in
the next table

\vspace{0.3cm}

\begin{center}
\begin{longtable}{|c|c|}
\hline 
Tango type & short cut (xxx)\tabularnewline
\hline 
\hline 
Tango::DevBoolean & Bo\tabularnewline
\hline 
Tango::DevShort & Sh\tabularnewline
\hline 
Tango::DevLong & Lg\tabularnewline
\hline 
Tango::DevFloat & Fl\tabularnewline
\hline 
Tango::DevDouble & Db\tabularnewline
\hline 
Tango::DevUshort & US\tabularnewline
\hline 
Tango::DevULong & UL\tabularnewline
\hline 
Tango::DevString & Str\tabularnewline
\hline 
Tango::DevVarCharArray & ChA\tabularnewline
\hline 
Tango::DevVarShortArray & ShA\tabularnewline
\hline 
Tango::DevVarLongArray & LgA\tabularnewline
\hline 
Tango::DevVarFloatArray & FlA\tabularnewline
\hline 
Tango::DevVarDoubleArray & DbA\tabularnewline
\hline 
Tango::DevVarUShortArray & USA\tabularnewline
\hline 
Tango::DevVarULongArray & ULA\tabularnewline
\hline 
Tango::DevVarStringArray & StrA\tabularnewline
\hline 
Tango::DevVarLongStringArray & LSA\tabularnewline
\hline 
Tango::DevVarDoubleStringArray & DSA\tabularnewline
\hline 
Tango::DevState & Sta\tabularnewline
\hline 
\end{longtable}
\par\end{center}

\vspace{0.3cm}

For instance, a pointer to a method which takes a Tango::DevVarStringArray
as input parameter must be statically casted to a Tango::CmdMethPtr\_StrA,
a pointer to a method which takes a Tango::DevLong data as input parameter
must be statically casted to a Tango::CmdMethPtr\_Lg.

The \textbf{Tango::xxx\_CmdMethPtr} is a pointer to a method of the
DeviceImpl class which returns data of one of the Tango type and has
no input parameter. xxx must be set according to the method return
data type following the same rules than those described in the previous
table. For instance, a pointer to a method which returns a Tango::DevDouble
data must be statically casted to a Tango::Db\_CmdMethPtr.

The \textbf{Tango::xxx\_CmdMethPtr\_yyy} is a pointer to a method
of the DeviceImpl class which returns data of one of the Tango type
and has one input parameter of one of the Tango data type. xxx and
yyy must be set according to the method return data type and parameter
type following the same rules than those described in the previous
table. For instance, a pointer to a method which returns a Tango::DevDouble
data and which takes a Tango::DevVarLongStringArray must be statically
casted to a Tango::Db\_CmdMethPtr\_LSA.

All those type are defined in the tango\_const.h file.

\subsection{Tango device state code}

The Tango::DevState type is a C++ enumeration starting at 0. The code
associated with each state\index{state} is defined in the following
table.

\vspace{0.3cm}

\begin{center}
\begin{longtable}{|c|c|}
\hline 
State name & Value\tabularnewline
\hline 
\hline 
Tango::ON & 0\tabularnewline
\hline 
Tango::OFF & 1\tabularnewline
\hline 
Tango::CLOSE & 2\tabularnewline
\hline 
Tango::OPEN & 3\tabularnewline
\hline 
Tango::INSERT & 4\tabularnewline
\hline 
Tango::EXTRACT & 5\tabularnewline
\hline 
Tango::MOVING & 6\tabularnewline
\hline 
Tango::STANDBY & 7\tabularnewline
\hline 
Tango::FAULT & 8\tabularnewline
\hline 
Tango::INIT & 9\tabularnewline
\hline 
Tango::RUNNING & 10\tabularnewline
\hline 
Tango::ALARM & 11\tabularnewline
\hline 
Tango::DISABLE & 12\tabularnewline
\hline 
Tango::UNKNOWN & 13\tabularnewline
\hline 
\end{longtable}
\par\end{center}

\vspace{0.3cm}

A strings array called \textbf{Tango::DevStateName\index{DevStateName}}
can be used to get the device state as a string. Use the Tango device
state code as index into the array to get the correct string.

\subsection{Tango data type }

A ``define'' has been created for each Tango data type. This is
summarized in the following table

\vspace{0.3cm}

\begin{center}
\begin{longtable}{|c|c|c|}
\hline 
Type name & Type code & Value\tabularnewline
\hline 
\hline 
Tango::DevBoolean & Tango::DEV\_BOOLEAN & 1\tabularnewline
\hline 
Tango::DevShort & Tango::DEV\_SHORT & 2\tabularnewline
\hline 
Tango::DevLong & Tango::DEV\_LONG & 3\tabularnewline
\hline 
Tango::DevFloat & Tango::DEV\_FLOAT & 4\tabularnewline
\hline 
Tango::DevDouble & Tango::DEV\_DOUBLE & 5\tabularnewline
\hline 
Tango::DevUShort & Tango::DEV\_USHORT & 6\tabularnewline
\hline 
Tango::DevULong & Tango::DEV\_ULONG & 7\tabularnewline
\hline 
Tango::DevString & Tango::DEV\_STRING & 8\tabularnewline
\hline 
Tango::DevVarCharArray & Tango::DEVVAR\_CHARARRAY & 9\tabularnewline
\hline 
Tango::DevVarShortArray & Tango::DEVVAR\_SHORTARRAY & 10\tabularnewline
\hline 
Tango::DevVarLongArray & Tango::DEVVAR\_LONGARRAY & 11\tabularnewline
\hline 
Tango::DevVarFloatArray & Tango::DEVVAR\_FLOATARRAY & 12\tabularnewline
\hline 
Tango::DevVarDoubleArray & Tango::DEVVAR\_DOUBLEARRAY & 13\tabularnewline
\hline 
Tango::DevVarUShortArray & Tango::DEVVAR\_USHORTARRAY & 14\tabularnewline
\hline 
Tango::DevVarULongArray & Tango::DEVVAR\_ULONGARRAY & 15\tabularnewline
\hline 
Tango::DevVarStringArray & Tango::DEVVAR\_STRINGARRAY & 16\tabularnewline
\hline 
Tango::DevVarLongStringArray & Tango::DEVVAR\_LONGSTRINGARRAY & 17\tabularnewline
\hline 
Tango::DevVarDoubleStringArray & Tango::DEVVAR\_DOUBLESTRINGARRAY & 18\tabularnewline
\hline 
Tango::DevState & Tango::DEV\_STATE & 19\tabularnewline
\hline 
Tango::ConstDevString & Tango::CONST\_DEV\_STRING & 20\tabularnewline
\hline 
Tango::DevVarBooleanArray & Tango::DEVVAR\_BOOLEANARRAY & 21\tabularnewline
\hline 
Tango::DevUChar & Tango::DEV\_UCHAR & 22\tabularnewline
\hline 
Tango::DevLong64 & Tango::DEV\_LONG64 & 23\tabularnewline
\hline 
Tango::DevULong64 & Tango::DEV\_ULONG64 & 24\tabularnewline
\hline 
Tango::DevVarLong64Array & Tango::DEVVAR\_LONG64ARRAY & 25\tabularnewline
\hline 
Tango::DevVarULong64Array & Tango::DEVVAR\_ULONG64ARRAY & 26\tabularnewline
\hline 
Tango::DevInt & Tango::DEV\_INT & 27\tabularnewline
\hline 
Tango::DevEncoded & Tango::DEV\_ENCODED & 28\tabularnewline
\hline 
Tango::DevEnum & Tango::DEV\_ENUM & 29\tabularnewline
\hline 
Tango::DevPipeBlob & Tango::DEV\_PIPE\_BLOB & 30\tabularnewline
\hline 
Tango::DevVarStateArray & Tango::DEVVAR\_STATEARRAY & 31\tabularnewline
\hline 
\end{longtable}
\par\end{center}

\vspace{0.3cm}

For command which do not take input parameter, the type code Tango::DEV\_VOID
(value = 0) has been defined.

A strings array called \textbf{Tango::CmdArgTypeName\index{CmdArgTypeName}}
can be used to get the data type as a string. Use the Tango data type
code as index into the array to get the correct string.

\subsection{Tango command display level}

Like attribute, Tango command has a display level. The Tango::DispLevel
type is a C++ enumeration starting at 0. The code associated with
each command display level is already described in page \pageref{display level}

As for attribute, this parameter allows a graphical application to
support two types of operation :
\begin{itemize}
\item An operator mode for day to day operation
\item An expert mode when tuning is necessary
\end{itemize}
According to this parameter, a graphical application knows if the
command is for the operator mode or for the expert mode.

\section{Device server process option and environment variables}

\subsection{Classical device server}

The synopsis of a device server process is\begin{center}ds\_name
instance\_name {[}OPTIONS{]}\end{center}The supported options are
:
\begin{itemize}
\item \textbf{-h, -? -help}\\
Print the device server synopsis and a list of instance name defined
in the database for this device server. An instance name in not mandatory
in the command line to use this option
\item \textbf{-v{[}trace level{]}}\\
Set the verbose level. If no trace level is given, a default value
of 4 is used
\item \textbf{-file=<file name path>}\\
Start a device server using an ASCII file instead of the Tango database. 
\item \textbf{-nodb}\\
Start a device server without using the database.
\item \textbf{-dlist <device name list>}\\
Give the device name list. This option is supported only with the
-nodb option.
\item \textbf{ORB options} (started with -ORBxxx)\\
Options directly passed to the underlying ORB. Should be rarely used
except the -ORBendPoint option for device server not using the database
\end{itemize}

\subsection{Device server process as Windows service}

When used as a Windows service\index{service}, a Tango device server
supports several new options. These options are :
\begin{itemize}
\item \textbf{-i}\\
Install the service
\item \textbf{-s}\\
Install the service and choose the automatic startup mode
\item \textbf{-u}\\
Un-install the service
\item \textbf{-dbg}\\
Run in console mode to debug service. The service must have been installed
prior to use it.
\end{itemize}
Note that these options must be used after the device server instance
name.

\subsection{Environment variables}

A few environment variables can be used to tune a Tango control system.
TANGO\_HOST\index{TANGO-HOST} is the most important one but on top
it, some Tango features like Tango logging service or controlled access
(if used) can be tuned using environment variable. If these environment
variables are not defined, the software searches in the file \textbf{\$HOME/.tangorc}
for its value. If the file is not defined or if the environment variable
is also not defined in this file, the software searches in the file
\textbf{/etc/tangorc}\index{tangorc} for its value. For Windows,
the file is \textbf{\$TANGO\_ROOT/tangorc} TANGO\_ROOT\index{TANGO-ROOT}
being the mandatory environment variable of the Windows binary distribution.

\subsubsection{TANGO\_HOST}

This environment variable is the anchor of the system. It specifies
where the Tango database server is running. Most of the time, its
syntax is\begin{center}TANGO\_HOST=<host>:<port>\end{center}host
is the name of the computer where the database server is running and
port is th eport number on which it is litenning. If you want to have
a Tango control system which has several database servers (but only
one database) in order to survive a database server crashes, use the
following syntax\begin{center}TANGO\_HOST=<host\_1>:<port\_1>,<host\_2>:<port\_2>,<host\_3>:<port\_3>\end{center}Obviously,
host\_1 is the name of the computer where the first database server
is running, port\_1 is the port number on which this server is listenning.
host\_2 is the name of the computer where the second database server
is running and port\_2 is its port number. All access to database
will automatically switch from one server to another one in the list
if the one which was used has died.

\subsubsection{Tango Logging Service (TANGO\_LOG\_PATH)}

The TANGO\_LOG\_PATH\index{TANGO-LOG-PATH} environment variable can
be used to specify the log files location. If not set it defaults
to /tmp/tango-<user logging name> under Unix and C:/tango-<user logging
name> under Windows. For a given device-server, the files are actually
saved into \$TANGO\_LOG\_PATH/\{ server\_name\}/\{ server\_instance\_name\}.
This means that all the devices running within the same process log
into the same directory. 

\subsubsection{The database and controlled access server (MYSQL\_USER, MYSQL\_PASSWORD,
MYSQL\_HOST and MYSQL\_DATABASE)\label{subsec:Db-Env-Variables}}

The Tango database server and the controlled access server (if used)
need to connect to the MySQL database. They are using four environment
variables called MYSQL\_USER\index{MYSQL-USER}, MYSQL\_PASSWORD\index{MYSQL-PASSWORD}
to know which user/password they must use to access the database,
MYSQL\_HOST\index{MYSQL-HOST} in case the MySQL database is running
on another host and MYSQL\_DATABASE\index{MYSQL-DATABASE} to specify
the name of the database to connect to. The MYSQL\_HOST environment
variable allows you to specify the host and port number where MySQL
is running. Its syntax is \begin{center}host:port\end{center} The
port definition is optional. If it is not specified, the default MySQL
port will be used. If these environment variables are not defined,
they will connect to the DBMS using the \textquotedbl{}root\textquotedbl{}
login on localhost with the MySQL default port number (3306). The
MYSQL\_DATABASE environment variable has to be used in case your are
using the same Tango Database device server executable code to connect
to several Tango databases each of them having a different name.

\subsubsection{The controlled access}

Even if a controlled access system is running, it is possible to by-pass
it if in the environment of the client application the environment
variable SUPER\_TANGO\index{SUPER-TANGO} is defined to \textquotedbl{}true\textquotedbl{}.

\subsubsection{The event buffer size}

If required, the event buffer used by the ZMQ\index{ZMQ} software
could be tuned using environment variables. These variables are named
TANGO\_DS\_EVENT\_BUFFER\_HWM\index{TANGO-DS-EVENT-BUFFER-HWM} for
the event buffer on a device server side and TANGO\_EVENT\_BUFFER\_HWM\index{TANGO-EVENT-BUFFER-HWM}
for the event buffer on the client size. Both of them are a number
which is the maximum number of events which could be stored in these
buffers.
