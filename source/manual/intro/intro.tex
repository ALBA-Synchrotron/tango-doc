
\chapter{Introduction}

\section{Introduction to device server}

Device servers were first developed at the European Synchrotron radiation
Facility (ESRF\index{ESRF}) for controlling the 6 Gev synchrotron
radiation source. This document is a Programmer's Manual on how to
write TANGO device servers. It will not go into the details of the
ESRF\index{ESRF}, nor its Control System nor any of the specific
device servers in the Control System. The role of this document is
to help programmers faced with the task of writing TANGO device servers.

Device servers have been developed at the ESRF\index{ESRF} in order
to solve the main task of Control Systems viz provide read and write
access to all devices in a distributed system. The problem of distributed
device access is only part of the problem however. The other part
of the problem is providing a programming framework for a large number
of devices programmed by a large number of programmers each having
different levels of experience and style.

Device servers have been written at the ESRF\index{ESRF} for a large
variety of different devices. Devices vary from serial line devices
to devices interfaced by field-bus to memory mapped VME cards or PC
cards to entire data acquisition systems. The definition of a device
depends very much on the user's requirements. In the simple case a
device server can be used to hide the serial line protocol required
to communicate with a device. For more complicated devices the device
server can be used to hide the entire complexity of the device timing,
configuration and acquisition cycle behind a set of high level commands.

In this manual the process of how to write TANGO client (applications)
and device servers will be treated. The manual has been organized
as follows :
\begin{itemize}
\item A getting started chapter. 
\item The TANGO device server model is treated in chapter 3 
\item Generalities on the Tango Application Programmer Interfaces are given
in chapter 4 
\item Chapter 5 is an a programmer's guide for the Tango Application ToolKit
(TangoATK). This is a Java toolkit to help Tango Java application
developers. 
\item How to write a TANGO device server is explained in chapter 6 
\item Chapter 7 describes advanced Tango features 
\end{itemize}
Throughout this manual examples of source code will be given in order
to illustrate what is meant. Most examples have been taken from the
StepperMotor class - a simulation of a stepper motor which illustrates
how a typical device server for a stepper motor at the ESRF functions.

\section{Device server history}

The concept of using device servers to access devices was first proposed
at the ESRF in 1989. It has been successfully used as the heart of
the ESRF\index{ESRF} Control System for the institute accelerator
complex. This Control System has been named TACO\index{TACO}\footnote{TACO stands for \textbf{T}elescope and \textbf{A}ccelerator \textbf{C}ontrolled
with \textbf{O}bjects}. Then, it has been decided to also used TACO to control devices in
the beam-lines. Today, more than 30 instances of TACO are running
at the ESRF. The main technologies used within TACO are the leading
technologies of the 80's. The Sun Remote Procedure Call (RPC) is used
to communicate over the network between device server and applications,
OS-9 is used on the front-end computers, C is the reference language
to write device servers and clients and the device server framework
follows the MIT Widget model. In 1999, a renewal of the control system
was started. In June 2002, Soleil\index{Soleil} and ESRF offically
decide to collaborate to develop this renewal of the old TACO control
system. Soleil is a French synchrotron radiation facility currently
under construction in the Paris suburbs. See \cite{Soleil_home_page}
to get all information about Soleil. In December 2003, Elettra\index{Elettra}
joins the club. Elettra is an Italian synchrotron radiation facility
located in Trieste. See \cite{Elettra_home_page} to get all information
about Elettra. Then, beginning of 2005, ALBA\index{ALBA} also decided
to join. ALBA is a Spanish synchrotron radiation facility located
in Barcelona. See \cite{Alba_WEB} to get all information about ALBA.
The new version of the Alba/Elettra/ESRF/Soleil control system is
named TANGO\footnote{TANGO stands for \textbf{TA}co \textbf{N}ext \textbf{G}eneration \textbf{O}bject}
and is based on the 21 century technologies :
\begin{itemize}
\item CORBA\index{CORBA}\footnote{CORBA stands for \textbf{C}ommon \textbf{O}bject \textbf{R}equest
\textbf{B}roker \textbf{A}rchitecture} and ZMQ\index{ZMQ}\cite{ZMQ} to communicate between device server
and clients 
\item C++, Python and Java as reference programming languages 
\item Linux and Windows as operating systems 
\item Modern object oriented design patterns
\end{itemize}

