
\chapter{Tango object naming (device, attribute and property)}


\section{Device name\label{DeviceNaming}}

A Tango device name is a three fields name. The field separator is
the \textbf{/} character. The first field is named \textbf{domain},
the second field is named \textbf{family} and the last field is named
\textbf{member}.A tango device name looks like \begin{center}domain/family/member\end{center} It
is a hierarchical notation. The member specifies which element within
a family. The family specifies which kind of equipment within a domain.
The domain groups devices related to which part of the accelerator/experiment
they belongs to. At ESRF, some of the machine control system domain
name are SR for the storage ring, TL1 for the transfer line 1 or SY
for the synchrotron booster. For experiment, ID11 is the domain name
for all devices belonging to the experiment behind insertion device
11. Here are some examples of Tango device name used at the ESRF :
\begin{itemize}
\item \textbf{sr/d-ct/1} : The current transformer. The domain part is sr
for storage ring. The family part is d-ct for diagnostic/current transformer
and the member part is 1
\item \textbf{fe/v-pen/id11-1} : A Penning gauge. The domain part is fe
for front-end. The family part is v-pen for vacuum/penning and the
member name is id11-1 to specify that this is the first gauge on the
front-end part after the insertion device 11
\end{itemize}

\section{Full object name}

The device name as described above is not enough to cover all Tango
usage like device server without database or device access for multi
control system. With the naming schema, we must also be able to name
attribute and property. Therefore, the full naming schema is\begin{center}\emph{{[}protocol://{]}{[}host:port/{]}device\_name{[}/attribute{]}{[}->property{]}{[}\#dbase=xx}]\end{center}The
protocol, host, port, attribute, property and dbase fields are optional.
The meaning of these fields are :
\begin{lyxlist}{00.00.0000}
\item [{protocol}] : Specifies which protocol is used (Tango or Taco).
Tango is the default
\item [{dbase=xx}] : The supported value for xx is \emph{yes} and \emph{no}.
This field is used to specify that the device is a device served by
a device server started with or without database usage. The default
value is \emph{dbase=yes}
\item [{host:port}] : This field has different meaning according to the
dbase value. If \emph{dbase=yes} (the default), the host is the host
where the control system database server is running and port is the
database server port. It has a higher priority than the value defined
by the TANGO\_HOST environment variable. If \emph{dbase=no}, host
is the host name where the device server process serving the device
is running and port is the device server process port.
\item [{attribute}] : The attribute name
\item [{property}] : The property name
\end{lyxlist}
The host:port and dbase=xx fields are necessary only when creating
the DeviceProxy object used to remotely access the device. The ->
characters are used to specify a property name.


\subsection{Some examples}


\subsubsection{Full device name examples}
\begin{itemize}
\item \textbf{gizmo:20000/sr/d-ct/1} : Device sr/d-ct/1 running in a specified
control system with the database server running on a host called gizmo
and using the port number 20000. The TANGO\_HOST environment variable
will not be used.
\item \textbf{tango://freak:2345/id11/rv/1\#dbase=no} : Device served by
a device server started without database. The server is running on
a host called freak and use port number 2345. //freak:2345/id11/rv/1\#dbase=no
is also possible for the same device.
\item \textbf{Taco://sy/ps-ki/1} : Taco device sy/ps-ki/1
\end{itemize}

\subsubsection{Attribute name examples}
\begin{itemize}
\item \textbf{id11/mot/1/Position} : Attribute position for device id11/mot/1 
\item \textbf{sr/d-ct/1/Lifetime} : Attribute lifetime for Tango device
sr/d-ct/1
\end{itemize}

\subsubsection{Attribute property name}
\begin{itemize}
\item \textbf{id11/rv/1/temp->label} : Property label for attribute temp
for device id11/rv/1.
\item \textbf{sr/d-ct/1/Lifetime->unit} : The unit property for the Lifetime
attribute of the sr/d-ct/1 device
\end{itemize}

\subsubsection{Device property name}
\begin{itemize}
\item \textbf{sr/d-ct/1->address} : the address property for device sr/d-ct/1
\end{itemize}

\subsubsection{Class property name}
\begin{itemize}
\item \textbf{Starter->doc\_url} : The doc\_url property for a class called
Starter
\end{itemize}

\section{Device and attribute name alias\index{alias}}

Within Tango, each device or attribute can have an alias name defined
in the database. Every time a device or an attribute name is requested
by the API's, it is possible to use the alias. The alias is simply
an open string stored in the database. The rule of the alias is to
give device or attribute name a name more natural from the physicist
point of view. Let's imagine that for experiment, the sample position
is described by angles called teta and psi in physics book. It is
more natural for physicist when they move the motor related to sample
position to use \emph{teta} and \emph{psi} rather device name like
\emph{idxx/mot/1} or \emph{idxx/mot/2}. An attribute alias is a synonym
for the four fields used to name an attribute. For instance, the attribute
\emph{Current} of a power-supply device called \emph{sr/ps/dipole}
could have an alias \emph{DipoleCurrent}. This alias can be used when
creating an instance of a AttributeProxy class instead of the full
attribute name which is \emph{sr/ps/dipole/Current}. Device alias
name are uniq within a Tango control system. Attribute alias name
are also uniq within a Tango control system.


\section{Reserved words and characters, limitations}

From the naming schema described above, the reserved characters are
\textbf{:},\textbf{\#},\textbf{/} and the reserved string is : \textbf{->}.
On top of that, the dbt\_update tool (tool to fulfill database from
the content of a file) reserved the \textbf{device} word

The device name, its domain, member and family fields and its alias
are stored in the Tango database. The default maximum size for these
items are :

\vspace{0.3cm}


\begin{center}
\begin{longtable}{|c|c|}
\hline 
Item & max length\tabularnewline
\hline 
\hline 
device name & 255\tabularnewline
\hline 
domain field & 85\tabularnewline
\hline 
family field & 85\tabularnewline
\hline 
member field & 85\tabularnewline
\hline 
device alias name & 255\tabularnewline
\hline 
\end{longtable}
\par\end{center}

\vspace{0.3cm}


The device name, the command name, the attribute name, the property
name, the device alias name and the device server name are \textbf{case
insensitive}.
