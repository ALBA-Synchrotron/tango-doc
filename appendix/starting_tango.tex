
\chapter{Starting a Tango control system}


\section{Without database}

When used without database\index{database}, there is no additional
process to start. Simply starts device server using the -nodb option
(and eventually the -dlist option) on specific port. See \ref{sec:Device-server-without}
to find informations on how to start/write Tango device server not
using the database.


\section{With database}

Starting the Tango control system simply means starting its database
device server on a well defined host using a well defined port. Use
the host name and the port number to build the TANGO\_HOST environment
variable. See \ref{Env variable} to find how starting a device server
on a specific host. Obviously, the underlying database software (MySQL)
must be started before the Tango database device server. The Tango
database server connects to MySQL using a default logging name set
to \textquotedbl{}root\textquotedbl{}. You can change this behaviour
with the MYSQL\_USER\index{MYSQL-USER} and MYSQL\_PASSWORD\index{MYSQL-PASSWORD}
environment variables. Define them before starting the database server.

If you are using the Tango administration graphical tool called \textbf{Astor}\index{Astor},
you also need to start a specific Tango device server called \textbf{Starter\index{Starter}}
on each host where Tango device server(s) are running. See \cite{Astor_doc}
for Astor documentation. This starter device server is able to start
even before the Tango database device server is started. In this case,
it will enter a loop in which it periodically tries to access the
Tango database device. The loop exits and the server starts only if
the database device access succeed.


\section{With database and event}


\subsection{For Tango releases lower than 8}

On top of what is described in the previous chapter, using event means
using CORBA Notification service. Start one Notification Service\index{Notification Service}
daemon on each host where device server(s) used via events are running.
The Notification Service daemon event channel factory IOR has to be
registered in the Tango database. This is done with the \textbf{notifd2db\index{notifd2db}}
command. The notification daemon is a process with a high thread number.
By default, Unix like operating systems reserve a big amount of memory
for each thread stack (8 MByte for Linux/Ubuntu, 10 MByte for Linux/RedHat
4). If your process has several hundreds of threads, this could generate
a too high memory requirement on virtual memory and even exceed the
maximun allowed memory per process (3 GBytes on Linux for 32 bits
computer). The notification service daemon works very well with a
value of only 2 Mybtes for thread stack. The Unix command line \textquotedbl{}ulimit\index{ulimit}
-s 2048\textquotedbl{} asks the operating system to give 2 Mbytes
for each thread stack. Example of starting and registering a Notification
Service daemon on a UNIX like operating system

%
% Copyright (C) :      2004,2005,2006,2007,2008,2009,2010,2011,2012,2013
%                      European Synchrotron Radiation Facility
%                      BP 220, Grenoble 38043
%                      FRANCE
%
% This file is part of Tango.
%
% Tango is free software: you can redistribute it and/or modify
% it under the terms of the GNU Lesser General Public License as published by
% the Free Software Foundation, either version 3 of the License, or
% (at your option) any later version.
%
% Tango is distributed in the hope that it will be useful,
% but WITHOUT ANY WARRANTY; without even the implied warranty of
% MERCHANTABILITY or FITNESS FOR A PARTICULAR PURPOSE.  See the
% GNU Lesser General Public License for more details.
%
% You should have received a copy of the GNU Lesser General Public License
% along with Tango.  If not, see <http://www.gnu.org/licenses/>.
%
\begin{flushleft}
\begin{picture}(0,0)
\thicklines
\put(0,0){\line(1,0){400}}
\end{picture}
\end{flushleft}

\begin{lyxcode}
~~~~~1~~ulimit~-s~2048

~~~~~2~~notifd~-n~-DDeadFilterInterval=300~\&

~~~~~3~~notifd2db
\end{lyxcode}
%
% Copyright (C) :      2004,2005,2006,2007,2008,2009,2010,2011,2012,2013
%                      European Synchrotron Radiation Facility
%                      BP 220, Grenoble 38043
%                      FRANCE
%
% This file is part of Tango.
%
% Tango is free software: you can redistribute it and/or modify
% it under the terms of the GNU Lesser General Public License as published by
% the Free Software Foundation, either version 3 of the License, or
% (at your option) any later version.
%
% Tango is distributed in the hope that it will be useful,
% but WITHOUT ANY WARRANTY; without even the implied warranty of
% MERCHANTABILITY or FITNESS FOR A PARTICULAR PURPOSE.  See the
% GNU Lesser General Public License for more details.
%
% You should have received a copy of the GNU Lesser General Public License
% along with Tango.  If not, see <http://www.gnu.org/licenses/>.
%
\begin{flushleft}
\begin{picture}(0,0)
\thicklines
\put(0,0){\line(1,0){400}}
\end{picture}
\end{flushleft}


The Notification Service daemon is started at line 2. Its \textquotedbl{}-DDeadFilterInterval\textquotedbl{}\index{DeadFilterInterval}
option is used to specify some internal cleaning of dead objects within
the notification service. The \textquotedbl{}-n\textquotedbl{} option
is used to disable the use of the CORBA Naming Service for registering
the default event channel factory. The registration of the Notification
Service daemon in the Tango database is done at line 2.

It differs on a Windows computer

%
% Copyright (C) :      2004,2005,2006,2007,2008,2009,2010,2011,2012,2013
%                      European Synchrotron Radiation Facility
%                      BP 220, Grenoble 38043
%                      FRANCE
%
% This file is part of Tango.
%
% Tango is free software: you can redistribute it and/or modify
% it under the terms of the GNU Lesser General Public License as published by
% the Free Software Foundation, either version 3 of the License, or
% (at your option) any later version.
%
% Tango is distributed in the hope that it will be useful,
% but WITHOUT ANY WARRANTY; without even the implied warranty of
% MERCHANTABILITY or FITNESS FOR A PARTICULAR PURPOSE.  See the
% GNU Lesser General Public License for more details.
%
% You should have received a copy of the GNU Lesser General Public License
% along with Tango.  If not, see <http://www.gnu.org/licenses/>.
%
\begin{flushleft}
\begin{picture}(0,0)
\thicklines
\put(0,0){\line(1,0){400}}
\end{picture}
\end{flushleft}

\begin{lyxcode}
~~~~~1~~notifd~-n~-DDeadFilterInterval=300~-DFactoryIORFileName=C:\textbackslash{}Temp\textbackslash{}evfact.ior~\&

~~~~~2~~notifd2db~C:\textbackslash{}Temp\textbackslash{}evfact.ior
\end{lyxcode}
%
% Copyright (C) :      2004,2005,2006,2007,2008,2009,2010,2011,2012,2013
%                      European Synchrotron Radiation Facility
%                      BP 220, Grenoble 38043
%                      FRANCE
%
% This file is part of Tango.
%
% Tango is free software: you can redistribute it and/or modify
% it under the terms of the GNU Lesser General Public License as published by
% the Free Software Foundation, either version 3 of the License, or
% (at your option) any later version.
%
% Tango is distributed in the hope that it will be useful,
% but WITHOUT ANY WARRANTY; without even the implied warranty of
% MERCHANTABILITY or FITNESS FOR A PARTICULAR PURPOSE.  See the
% GNU Lesser General Public License for more details.
%
% You should have received a copy of the GNU Lesser General Public License
% along with Tango.  If not, see <http://www.gnu.org/licenses/>.
%
\begin{flushleft}
\begin{picture}(0,0)
\thicklines
\put(0,0){\line(1,0){400}}
\end{picture}
\end{flushleft}



\subsection{For release 8 and above}

A new event system has been implemented starting with Tango release
8. With this new event system, the CORBA Notification service is not
needed any more. This means that \textbf{as soon} as all Tango device
server processes running on a host and all clients using events from
their devices used Tango 8, it is not required to start any process
other than the device servers and the clients. You can forget the
previous sub-chapter!


\section{With file used as database}

When used with database\index{database} on file, there is no additional
process to start. Simply starts device server using the -file option
specifying file name port. See \ref{sec:Device-server-file} to find
informations on how to start Tango device server using database on
file.


\section{With file used as database and event}


\subsection{For Tango releases lower than 8}

Using event means using CORBA Notification service. Start one Notification
Service\index{Notification Service} daemon on the host where device
server(s) using events are running. The Notification Service daemon
event channel factory IOR has to be registered in the file(s) use
as database. This is done with the \textbf{notifd2db\index{notifd2db}}
command. Example of starting and registering a Notification Service
daemon on a UNIX like operating system

%
% Copyright (C) :      2004,2005,2006,2007,2008,2009,2010,2011,2012,2013
%                      European Synchrotron Radiation Facility
%                      BP 220, Grenoble 38043
%                      FRANCE
%
% This file is part of Tango.
%
% Tango is free software: you can redistribute it and/or modify
% it under the terms of the GNU Lesser General Public License as published by
% the Free Software Foundation, either version 3 of the License, or
% (at your option) any later version.
%
% Tango is distributed in the hope that it will be useful,
% but WITHOUT ANY WARRANTY; without even the implied warranty of
% MERCHANTABILITY or FITNESS FOR A PARTICULAR PURPOSE.  See the
% GNU Lesser General Public License for more details.
%
% You should have received a copy of the GNU Lesser General Public License
% along with Tango.  If not, see <http://www.gnu.org/licenses/>.
%
\begin{flushleft}
\begin{picture}(0,0)
\thicklines
\put(0,0){\line(1,0){400}}
\end{picture}
\end{flushleft}

\begin{lyxcode}
~~~~~1~~notifd~-n~-DDeadFilterInterval=300~\&

~~~~~2~~notifd2db~-o~/var/myfile.res
\end{lyxcode}
%
% Copyright (C) :      2004,2005,2006,2007,2008,2009,2010,2011,2012,2013
%                      European Synchrotron Radiation Facility
%                      BP 220, Grenoble 38043
%                      FRANCE
%
% This file is part of Tango.
%
% Tango is free software: you can redistribute it and/or modify
% it under the terms of the GNU Lesser General Public License as published by
% the Free Software Foundation, either version 3 of the License, or
% (at your option) any later version.
%
% Tango is distributed in the hope that it will be useful,
% but WITHOUT ANY WARRANTY; without even the implied warranty of
% MERCHANTABILITY or FITNESS FOR A PARTICULAR PURPOSE.  See the
% GNU Lesser General Public License for more details.
%
% You should have received a copy of the GNU Lesser General Public License
% along with Tango.  If not, see <http://www.gnu.org/licenses/>.
%
\begin{flushleft}
\begin{picture}(0,0)
\thicklines
\put(0,0){\line(1,0){400}}
\end{picture}
\end{flushleft}


The Notification Service daemon is started at line 1. Its \textquotedbl{}-n\textquotedbl{}
option is used to disable the use of the CORBA Naming Service for
registering the default event channel factory. The registration of
the Notification Service daemon in the file used as database is done
at line 2 with its \textbf{-o} command line option.

It differs on a Windows computer because the name of the file used
by the CORBA notification service to store its channel factory IOR
must be specified using its -D command line option. This file name
has also to be passed to the notifd2db command.

%
% Copyright (C) :      2004,2005,2006,2007,2008,2009,2010,2011,2012,2013
%                      European Synchrotron Radiation Facility
%                      BP 220, Grenoble 38043
%                      FRANCE
%
% This file is part of Tango.
%
% Tango is free software: you can redistribute it and/or modify
% it under the terms of the GNU Lesser General Public License as published by
% the Free Software Foundation, either version 3 of the License, or
% (at your option) any later version.
%
% Tango is distributed in the hope that it will be useful,
% but WITHOUT ANY WARRANTY; without even the implied warranty of
% MERCHANTABILITY or FITNESS FOR A PARTICULAR PURPOSE.  See the
% GNU Lesser General Public License for more details.
%
% You should have received a copy of the GNU Lesser General Public License
% along with Tango.  If not, see <http://www.gnu.org/licenses/>.
%
\begin{flushleft}
\begin{picture}(0,0)
\thicklines
\put(0,0){\line(1,0){400}}
\end{picture}
\end{flushleft}

\begin{lyxcode}
~~~~~1~~notifd~-n~-DDeadFilterInterval=300~-DFactoryIORFileName=C:\textbackslash{}Temp\textbackslash{}evfact.ior~\&

~~~~~2~~notifd2db~C:\textbackslash{}Temp\textbackslash{}evfact.ior~-o~C:\textbackslash{}Temp\textbackslash{}myfile.res
\end{lyxcode}
%
% Copyright (C) :      2004,2005,2006,2007,2008,2009,2010,2011,2012,2013
%                      European Synchrotron Radiation Facility
%                      BP 220, Grenoble 38043
%                      FRANCE
%
% This file is part of Tango.
%
% Tango is free software: you can redistribute it and/or modify
% it under the terms of the GNU Lesser General Public License as published by
% the Free Software Foundation, either version 3 of the License, or
% (at your option) any later version.
%
% Tango is distributed in the hope that it will be useful,
% but WITHOUT ANY WARRANTY; without even the implied warranty of
% MERCHANTABILITY or FITNESS FOR A PARTICULAR PURPOSE.  See the
% GNU Lesser General Public License for more details.
%
% You should have received a copy of the GNU Lesser General Public License
% along with Tango.  If not, see <http://www.gnu.org/licenses/>.
%
\begin{flushleft}
\begin{picture}(0,0)
\thicklines
\put(0,0){\line(1,0){400}}
\end{picture}
\end{flushleft}



\subsection{For release 8 and above}

A new event system has been implemented starting with Tango release
8. With this new event system, the CORBA Notification service is not
needed any more. This means that \textbf{as soon} as all clients using
events from devices embedded in the device server use Tango 8, it
is not required to start any process other than the device server
and its clients.


\section{With the controlled access}

Using the Tango controlled access means starting a specific device
server called TangoAccessControl\index{TangoAccessControl}. By default,
this server has to be started with the instance name set to \textquotedbl{}1\textquotedbl{}
and its device name is \textquotedbl{}sys/access\_control/1\textquotedbl{}.
The command line to start this device server is:
\begin{lyxcode}
TangoAccessControl~1
\end{lyxcode}
This server connects to MySQL using a default logging name set to
\textquotedbl{}root\textquotedbl{}. You can change this behaviour
with the MYSQL\_USER\index{MYSQL-USER} and MYSQL\_PASSWORD\index{MYSQL-PASSWORD}
environment variables. Define them before starting the controlled
access device server. This server also uses the MYSQL\_HOST\index{MYSQL-HOST}
environment variable if you need to connect it to some MySQL server
running on another host. The syntax of this environment varaible is
\textquotedbl{}host:port\textquotedbl{}. Port is optional and if it
is not defined, the MySQL default port is used (3306). If it is not
defined at all, a connection to the localhost is made. This controlled
access system uses the Tango database to retrieve user rights and
it is not possible to run it in a Tango control system running without
database.
